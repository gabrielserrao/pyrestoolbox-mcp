\documentclass[11pt,a4paper]{article}

% Packages
\usepackage[utf8]{inputenc}
\usepackage[T1]{fontenc}
\usepackage{amsmath}
\usepackage{amssymb}
\usepackage{geometry}
\usepackage{graphicx}
\usepackage{booktabs}
\usepackage{xcolor}
\usepackage{hyperref}
\usepackage{fancyhdr}
\usepackage{tcolorbox}
\usepackage{enumitem}
\usepackage{array}
\usepackage{longtable}
\usepackage{mathtools}

% Page geometry
\geometry{
    a4paper,
    left=2.5cm,
    right=2.5cm,
    top=3cm,
    bottom=3cm
}

% Header and footer
\pagestyle{fancy}
\fancyhf{}
\fancyhead[L]{\textit{pyResToolbox MCP}}
\fancyhead[R]{\textit{Formula Reference Guide}}
\fancyfoot[C]{\thepage}

% Hyperref setup
\hypersetup{
    colorlinks=true,
    linkcolor=blue,
    filecolor=magenta,
    urlcolor=cyan,
    pdftitle={pyResToolbox MCP Formula Reference},
    pdfauthor={Gabriel Serrao},
}

% Custom colors
\definecolor{sectioncolor}{RGB}{0,102,204}
\definecolor{subsectioncolor}{RGB}{51,153,255}
\definecolor{formulabox}{RGB}{245,245,250}

% Custom commands
\newcommand{\unit}[1]{\ensuremath{\,\text{#1}}}
\newcommand{\sym}[2]{\textbf{#1} & #2 \\}
\newcommand{\symitem}[2]{\item \textbf{#1}: #2}

% Title information
\title{
    \vspace{-2cm}
    \Huge\textbf{pyResToolbox MCP} \\
    \LARGE Formula Reference Guide \\
    \vspace{0.5cm}
    \large Quick Reference for Reservoir Engineering Calculations
}
\author{Gabriel Serrão Seabra}
\date{\today \\ \vspace{0.5cm} Version 1.0}

\begin{document}

\maketitle

\begin{abstract}
This document provides a comprehensive reference guide for calculation tools available in the pyResToolbox Model Context Protocol (MCP) server. It includes mathematical formulas, symbol definitions, unit systems, and correlation methods covering oil PVT, gas PVT, well performance, simulation support, brine properties, reservoir heterogeneity analysis, and geomechanics/wellbore stability. All calculations use Field Units (US Oilfield standard).

\vspace{0.5cm}

\textbf{Based on pyResToolbox:} This MCP server is built upon the excellent \textit{pyResToolbox} library by Mark W. Burgoyne (\url{https://github.com/mwburgoyne/pyResToolbox}), which provides the underlying reservoir engineering calculations.

\vspace{0.5cm}

\begin{tcolorbox}[colback=gray!5!white,colframe=black!75,title=\textbf{DISCLAIMER}]
\textbf{No Warranty:} This document and the formulas contained herein are provided "AS IS" without warranty of any kind, either expressed or implied. While every effort has been made to ensure accuracy, all formulas, correlations, and calculations \textbf{MUST be independently verified} before use in any engineering or commercial application.

\textbf{Professional Responsibility:} Users are solely responsible for verifying the correctness and applicability of any formula, correlation, or calculation method presented in this document. The author and contributors assume no liability for errors, omissions, or consequences arising from the use of this information.

\textbf{Verification Required:} Always cross-reference formulas with original published sources and validate results against known benchmarks before applying to real reservoir engineering problems.
\end{tcolorbox}
\end{abstract}

\newpage

\tableofcontents
\newpage

%=============================================================================
\section{Introduction}
%=============================================================================

\subsection{About pyResToolbox MCP}
pyResToolbox MCP is a Model Context Protocol server that provides AI agents like Claude with access to industry-standard petroleum engineering calculations. Built on the comprehensive pyResToolbox library by Mark Burgoyne, it offers 47 production-ready tools for reservoir engineering workflows.

\subsection{Unit System}
\textbf{All calculations use Field Units (US Oilfield standard):}

\begin{table}[h]
\centering
\begin{tabular}{lll}
\toprule
\textbf{Property} & \textbf{Unit} & \textbf{Example} \\
\midrule
Pressure & psia & 3000 psia \\
Temperature & °F & 180°F \\
Permeability & mD & 100 mD \\
Pay Thickness & ft & 50 ft \\
Viscosity & cP & 0.85 cP \\
Oil Rate & STB/day & 542 STB/day \\
Gas Rate & MSCF/day & 1250 MSCF/day \\
Oil Gravity & API° or SG & 35° API \\
Gas Gravity & SG (air=1) & 0.75 \\
Solution GOR & scf/stb & 800 scf/stb \\
Oil FVF & rb/stb & 1.25 rb/stb \\
Gas FVF & rcf/scf & 0.0045 rcf/scf \\
Compressibility & 1/psi & $1.2 \times 10^{-5}$ 1/psi \\
Density & lb/ft³ & 42.5 lb/ft³ \\
\bottomrule
\end{tabular}
\caption{Field Units Used in All Calculations}
\end{table}

\newpage

%=============================================================================
\section{Oil PVT Calculations}
%=============================================================================

\subsection{Bubble Point Pressure ($P_b$)}

\subsubsection{Description}
The bubble point pressure is the pressure at which the first gas bubble evolves from solution oil. It is a critical PVT property that defines the boundary between undersaturated (single-phase) and saturated (two-phase) oil behavior.

\subsubsection{Available Methods}

\textbf{1. Standing (1947) - STAN}

\begin{tcolorbox}[colback=formulabox,colframe=black!75]
\begin{equation}
P_b = 18.2 \left[ \left( \frac{R_{sb}}{\gamma_g} \right)^{0.83} \times 10^{a} - 1.4 \right]
\end{equation}
where:
\begin{equation}
a = 0.00091 \times T - 0.0125 \times \text{API}
\end{equation}
\end{tcolorbox}

\textbf{Symbol Definitions:}
\begin{itemize}[leftmargin=2cm]
\symitem{$P_b$}{Bubble point pressure (psia)}
\symitem{$R_{sb}$}{Solution GOR at bubble point (scf/stb)}
\symitem{$\gamma_g$}{Gas specific gravity (air=1)}
\symitem{$T$}{Temperature (°F)}
\symitem{$\text{API}$}{Oil API gravity (degrees)}
\end{itemize}

\textbf{2. Valko-McCain (2003) - VALMC}

\begin{tcolorbox}[colback=formulabox,colframe=black!75]
\begin{equation}
P_b = -C_1 - C_2 \times A - C_3 \times A^2
\end{equation}
where:
\begin{align}
A &= \ln \left( \frac{R_{sb}}{\gamma_{sp}} \right) \\
C_1 &= 7.916 \times 10^{-4} \times \text{API}^{1.541} - 4.561 \times 10^{-5} \times T^{1.3911} \\
C_2 &= 4.51 \times 10^{-5} \times \text{API}^{1.3911} \times T^{0.7849} \\
C_3 &= 1.046 \times 10^{-7} \times \text{API}^{1.3911} \times T^{1.007}
\end{align}
\end{tcolorbox}

\textbf{Additional Symbols:}
\begin{itemize}[leftmargin=2cm]
\symitem{$\gamma_{sp}$}{Separator gas specific gravity (air=1)}
\symitem{$C_1, C_2, C_3$}{Correlation coefficients}
\end{itemize}

\textbf{3. Velarde-Blasingame-McCain (1997) - VELAR}

\begin{tcolorbox}[colback=formulabox,colframe=black!75]
\begin{equation}
\ln(P_b) = a_0 + a_1 \ln(R_{sb}) + a_2 [\ln(R_{sb})]^2 + a_3 \ln(\gamma_{sp}) + a_4 \ln(\text{API}) + a_5 \ln(T+460)
\end{equation}
\end{tcolorbox}

Coefficients $a_0$ through $a_5$ are empirically determined from regression analysis.

\subsection{Solution Gas-Oil Ratio ($R_s$)}

\subsubsection{Description}
The solution gas-oil ratio represents the volume of gas dissolved in oil at a given pressure and temperature. It increases with pressure up to the bubble point, then remains constant (equal to $R_{sb}$) above the bubble point.

\subsubsection{Standing (1947) Method}

\begin{tcolorbox}[colback=formulabox,colframe=black!75]
\begin{equation}
R_s = \gamma_g \left[ \frac{P}{18.2} + 1.4 \right] \times 10^{-a}
\end{equation}
where:
\begin{equation}
a = 0.0125 \times \text{API} - 0.00091 \times T
\end{equation}
\end{tcolorbox}

\textbf{Symbol Definitions:}
\begin{itemize}[leftmargin=2cm]
\symitem{$R_s$}{Solution gas-oil ratio (scf/stb)}
\symitem{$P$}{Pressure (psia)}
\symitem{$\gamma_g$}{Gas specific gravity (air=1)}
\symitem{$\text{API}$}{Oil API gravity (degrees)}
\symitem{$T$}{Temperature (°F)}
\end{itemize}

\textbf{Behavior:}
\begin{itemize}
\item $P < P_b$: $R_s$ calculated from correlation
\item $P \geq P_b$: $R_s = R_{sb}$ (constant)
\end{itemize}

\subsection{Oil Formation Volume Factor ($B_o$)}

\subsubsection{Description}
The formation volume factor relates the volume of oil at reservoir conditions to its volume at stock tank conditions. It accounts for thermal expansion and gas in solution.

\subsubsection{McCain et al. (1988) Method - MCAIN}

\begin{tcolorbox}[colback=formulabox,colframe=black!75]
\begin{equation}
B_o = \frac{\rho_{ST}}{\rho_o}
\end{equation}
where oil density $\rho_o$ is calculated from:
\begin{equation}
\rho_o = \frac{\rho_{ST} + 0.01357 \times R_s \times \gamma_g}{0.972 + 0.000147 \times (T+460)^{1.175}}
\end{equation}
\end{tcolorbox}

\textbf{Symbol Definitions:}
\begin{itemize}[leftmargin=2cm]
\symitem{$B_o$}{Oil formation volume factor (rb/stb)}
\symitem{$\rho_o$}{Oil density at reservoir conditions (lb/ft³)}
\symitem{$\rho_{ST}$}{Oil density at stock tank conditions (lb/ft³)}
\symitem{$R_s$}{Solution gas-oil ratio (scf/stb)}
\symitem{$\gamma_g$}{Gas specific gravity (air=1)}
\symitem{$T$}{Temperature (°F)}
\end{itemize}

\subsubsection{Standing (1947) Method - STAN}

\begin{tcolorbox}[colback=formulabox,colframe=black!75]
\begin{equation}
B_o = 0.9759 + 0.00012 \left[ R_s \sqrt{\frac{\gamma_g}{\gamma_o}} + 1.25 T \right]^{1.2}
\end{equation}
\end{tcolorbox}

\textbf{Additional Symbols:}
\begin{itemize}[leftmargin=2cm]
\symitem{$\gamma_o$}{Oil specific gravity (water=1)}
\end{itemize}

\subsection{Oil Viscosity ($\mu_o$)}

\subsubsection{Description}
Oil viscosity is a measure of the oil's resistance to flow. It decreases with increasing temperature and decreasing pressure (above bubble point).

\subsubsection{Beggs-Robinson (1975) Method - BR}

\textbf{Dead Oil Viscosity:}
\begin{tcolorbox}[colback=formulabox,colframe=black!75]
\begin{equation}
\mu_{od} = 10^X - 1
\end{equation}
where:
\begin{equation}
X = T^{-1.163} \times \exp(6.9824 - 0.04658 \times \text{API})
\end{equation}
\end{tcolorbox}

\textbf{Live Oil Viscosity:}
\begin{tcolorbox}[colback=formulabox,colframe=black!75]
\begin{equation}
\mu_o = A \times \mu_{od}^B
\end{equation}
where:
\begin{align}
A &= 10.715 \times (R_s + 100)^{-0.515} \\
B &= 5.44 \times (R_s + 150)^{-0.338}
\end{align}
\end{tcolorbox}

\textbf{Symbol Definitions:}
\begin{itemize}[leftmargin=2cm]
\symitem{$\mu_o$}{Oil viscosity (cP)}
\symitem{$\mu_{od}$}{Dead oil viscosity (cP)}
\symitem{$R_s$}{Solution gas-oil ratio (scf/stb)}
\symitem{$T$}{Temperature (°F)}
\symitem{$\text{API}$}{Oil API gravity (degrees)}
\end{itemize}

\subsection{Oil Density ($\rho_o$)}

\subsubsection{Description}
Oil density at reservoir conditions accounts for dissolved gas and thermal expansion effects.

\subsubsection{Standing-White-McCain-Hill (1995) Method - SWMH}

\begin{tcolorbox}[colback=formulabox,colframe=black!75]
\begin{equation}
\rho_o = \frac{62.4 \times \gamma_o + 0.0136 \times R_s \times \gamma_g}{B_o}
\end{equation}
\end{tcolorbox}

\textbf{Symbol Definitions:}
\begin{itemize}[leftmargin=2cm]
\symitem{$\rho_o$}{Oil density (lb/ft³)}
\symitem{$\gamma_o$}{Oil specific gravity (water=1)}
\symitem{$\gamma_g$}{Gas specific gravity (air=1)}
\symitem{$R_s$}{Solution gas-oil ratio (scf/stb)}
\symitem{$B_o$}{Oil formation volume factor (rb/stb)}
\end{itemize}

\subsection{Oil Compressibility ($c_o$)}

\subsubsection{Description}
Oil compressibility describes the fractional change in oil volume with pressure. It is critical for material balance calculations and pressure transient analysis.

\subsubsection{Explicit Method - EXPLT}

\begin{tcolorbox}[colback=formulabox,colframe=black!75]
\begin{equation}
c_o = -\frac{1}{B_o} \left[ \frac{dB_o}{dP} - \frac{B_g}{5.61458} \frac{dR_s}{dP} \right]
\end{equation}
\end{tcolorbox}

\textbf{Symbol Definitions:}
\begin{itemize}[leftmargin=2cm]
\symitem{$c_o$}{Oil compressibility (1/psi)}
\symitem{$B_o$}{Oil formation volume factor (rb/stb)}
\symitem{$B_g$}{Gas formation volume factor (rcf/scf)}
\symitem{$R_s$}{Solution gas-oil ratio (scf/stb)}
\symitem{$P$}{Pressure (psia)}
\end{itemize}

\textbf{Note:} Derivatives are calculated numerically using finite differences.

\subsection{API-Specific Gravity Conversions}

\subsubsection{API to Specific Gravity}

\begin{tcolorbox}[colback=formulabox,colframe=black!75]
\begin{equation}
\gamma_o = \frac{141.5}{\text{API} + 131.5}
\end{equation}
\end{tcolorbox}

\subsubsection{Specific Gravity to API}

\begin{tcolorbox}[colback=formulabox,colframe=black!75]
\begin{equation}
\text{API} = \frac{141.5}{\gamma_o} - 131.5
\end{equation}
\end{tcolorbox}

\textbf{Symbol Definitions:}
\begin{itemize}[leftmargin=2cm]
\symitem{$\text{API}$}{Oil API gravity (degrees)}
\symitem{$\gamma_o$}{Oil specific gravity (water=1)}
\end{itemize}

\newpage

%=============================================================================
\section{Gas PVT Calculations}
%=============================================================================

\subsection{Gas Z-Factor (Compressibility Factor)}

\subsubsection{Description}
The Z-factor corrects the ideal gas law for real gas behavior. It is defined as:
\begin{equation}
PV = ZnRT
\end{equation}
where Z = 1 for ideal gas, Z < 1 for most real gases at reservoir conditions.

\subsubsection{Dranchuk-Abou-Kassem (1975) Method - DAK}

\begin{tcolorbox}[colback=formulabox,colframe=black!75]
\begin{align}
Z &= 1 + A_1 \rho_r + A_2 \rho_r^2 + A_3 \rho_r^5 + A_4 \rho_r^2 (1 + A_5 \rho_r^2) \exp(-A_5 \rho_r^2) \\
&\quad + \frac{A_6 + A_7 \rho_r + A_8 \rho_r^2}{T_{pr}^3} \rho_r^2
\end{align}

where reduced density $\rho_r$ is solved iteratively from:
\begin{equation}
\rho_r = \frac{0.27 P_{pr}}{Z T_{pr}}
\end{equation}
\end{tcolorbox}

\textbf{Symbol Definitions:}
\begin{itemize}[leftmargin=2cm]
\symitem{$Z$}{Gas compressibility factor (dimensionless)}
\symitem{$\rho_r$}{Reduced density (dimensionless)}
\symitem{$P_{pr}$}{Pseudo-reduced pressure = $P/P_c$}
\symitem{$T_{pr}$}{Pseudo-reduced temperature = $(T+460)/T_c$}
\symitem{$P_c$}{Pseudo-critical pressure (psia)}
\symitem{$T_c$}{Pseudo-critical temperature (°R)}
\symitem{$A_1, \ldots, A_8$}{Correlation constants}
\end{itemize}

\subsubsection{Hall-Yarborough (1973) Method - HY}

\begin{tcolorbox}[colback=formulabox,colframe=black!75]
Implicit equation for reduced density $y$:
\begin{equation}
\frac{0.06125 P_{pr} t}{y} \exp\left[-1.2(1-t)^2\right] = F(y)
\end{equation}
where:
\begin{equation}
F(y) = -A_1 + A_2/y + A_3/y^2 + A_4/y^3 + A_5/y^4 + y^2(A_6 + A_7/y + A_8/y^2)(1 + A_9y^2)\exp(-A_9y^2)
\end{equation}

Then Z is calculated from:
\begin{equation}
Z = \frac{A_1 P_{pr}}{y}
\end{equation}

where $t = 1/T_{pr}$ and $A_1, \ldots, A_9$ are correlation constants.
\end{tcolorbox}

\subsubsection{Burgoyne-Nielsen-Stanko (2025) Method - BUR}

\begin{tcolorbox}[colback=formulabox,colframe=black!75]
\textbf{Universal EOS-based correlation} for Z-factor, viscosity, and enthalpy. This is the most recent and comprehensive method, developed specifically to handle:
\begin{itemize}
\item High concentrations of non-hydrocarbons (CO$_2$, H$_2$S, N$_2$, H$_2$)
\item Pure non-hydrocarbon streams (100\% CO$_2$, 100\% H$_2$)
\item Hydrogen mixtures (only method supporting H$_2$)
\item Full range of P-T conditions
\end{itemize}

Based on a tuned 5-component Peng-Robinson EOS model. The formulation methods are fully described in SPE-229932-MS, and the tuning datasets are available on GitHub. The correlation is implemented in pyResToolbox and provides open access to these calculations.

\textbf{Reference:} SPE-229932-MS (ADIPEC 2025) - Burgoyne, M.W. (Santos Ltd), Nielsen, M.H., and Stanko, M. (Whitson AS)
\end{tcolorbox}

\textbf{Method Comparison:}
\begin{itemize}
\item \textbf{DAK}: Best for hydrocarbon gases with low non-HC content
\item \textbf{HY}: Fast alternative for standard conditions
\item \textbf{WYW}: Modern update to classical methods
\item \textbf{BUR}: Universal method, required for H$_2$ or high non-HC content
\end{itemize}

\subsection{Pseudo-Critical Properties}

\subsubsection{Description}
Pseudo-critical properties (temperature and pressure) are used to calculate the reduced properties needed for Z-factor correlations.

\subsubsection{Piper-McCain-Corredor (1993) Method - PMC}

\begin{tcolorbox}[colback=formulabox,colframe=black!75]
\textbf{Without Non-Hydrocarbons:}
\begin{align}
T_c &= 169.2 + 349.5 \gamma_g - 74.0 \gamma_g^2 \quad \text{(°R)} \\
P_c &= 756.8 - 131.07 \gamma_g - 3.6 \gamma_g^2 \quad \text{(psia)}
\end{align}

\textbf{With CO$_2$ and H$_2$S Corrections:}
\begin{align}
T_c' &= T_c - \epsilon_T \\
P_c' &= \frac{P_c - \epsilon_P}{1 - y_{CO_2} - y_{H_2S}}
\end{align}
where:
\begin{align}
\epsilon_T &= 120(y_{CO_2}^{0.9} - y_{CO_2}^{1.6}) + 15(y_{H_2S}^{0.5} - y_{H_2S}^4) \\
\epsilon_P &= 440 y_{CO_2} + 600 y_{H_2S} - 170 y_{CO_2} y_{H_2S}
\end{align}
\end{tcolorbox}

\textbf{Symbol Definitions:}
\begin{itemize}[leftmargin=2cm]
\symitem{$T_c$}{Pseudo-critical temperature (°R)}
\symitem{$P_c$}{Pseudo-critical pressure (psia)}
\symitem{$\gamma_g$}{Gas specific gravity (air=1)}
\symitem{$y_{CO_2}$}{CO$_2$ mole fraction}
\symitem{$y_{H_2S}$}{H$_2$S mole fraction}
\symitem{$y_{N_2}$}{N$_2$ mole fraction}
\symitem{$\epsilon_T$}{Temperature correction}
\symitem{$\epsilon_P$}{Pressure correction}
\end{itemize}

\subsection{Gas Formation Volume Factor ($B_g$)}

\subsubsection{Description}
The gas formation volume factor relates gas volume at reservoir conditions to its volume at standard conditions.

\begin{tcolorbox}[colback=formulabox,colframe=black!75]
\begin{equation}
B_g = \frac{0.02827 \times Z \times (T + 460)}{P} \quad \text{(rcf/scf)}
\end{equation}
\end{tcolorbox}

\textbf{Symbol Definitions:}
\begin{itemize}[leftmargin=2cm]
\symitem{$B_g$}{Gas formation volume factor (rcf/scf)}
\symitem{$Z$}{Gas compressibility factor (dimensionless)}
\symitem{$T$}{Temperature (°F)}
\symitem{$P$}{Pressure (psia)}
\end{itemize}

\subsection{Gas Viscosity ($\mu_g$)}

\subsubsection{Description}
Gas viscosity is required for flow calculations and increases with pressure and temperature.

\subsubsection{Lee-Gonzalez-Eakin (1966) Method - LGE}

\begin{tcolorbox}[colback=formulabox,colframe=black!75]
\begin{equation}
\mu_g = K \times 10^{-4} \times \exp\left[X \times \rho_g^Y\right]
\end{equation}
where:
\begin{align}
K &= \frac{(9.4 + 0.02 M) (T+460)^{1.5}}{209 + 19M + (T+460)} \\
X &= 3.5 + \frac{986}{T+460} + 0.01M \\
Y &= 2.4 - 0.2X \\
M &= 28.97 \gamma_g
\end{align}
\end{tcolorbox}

\textbf{Symbol Definitions:}
\begin{itemize}[leftmargin=2cm]
\symitem{$\mu_g$}{Gas viscosity (cP)}
\symitem{$\rho_g$}{Gas density (lb/ft³)}
\symitem{$M$}{Gas molecular weight (lb/lbmol)}
\symitem{$\gamma_g$}{Gas specific gravity (air=1)}
\symitem{$T$}{Temperature (°F)}
\end{itemize}

\subsection{Gas Density ($\rho_g$)}

\begin{tcolorbox}[colback=formulabox,colframe=black!75]
\begin{equation}
\rho_g = \frac{P \times 28.97 \times \gamma_g}{Z \times 10.73 \times (T+460)} \quad \text{(lb/ft³)}
\end{equation}
\end{tcolorbox}

\textbf{Symbol Definitions:}
\begin{itemize}[leftmargin=2cm]
\symitem{$\rho_g$}{Gas density (lb/ft³)}
\symitem{$P$}{Pressure (psia)}
\symitem{$\gamma_g$}{Gas specific gravity (air=1)}
\symitem{$Z$}{Gas compressibility factor (dimensionless)}
\symitem{$T$}{Temperature (°F)}
\end{itemize}

\subsection{Gas Compressibility ($c_g$)}

\subsubsection{Description}
Gas compressibility describes the fractional change in gas volume with pressure.

\begin{tcolorbox}[colback=formulabox,colframe=black!75]
\begin{equation}
c_g = \frac{1}{P} - \frac{1}{Z} \frac{dZ}{dP}
\end{equation}
\end{tcolorbox}

\textbf{Symbol Definitions:}
\begin{itemize}[leftmargin=2cm]
\symitem{$c_g$}{Gas compressibility (1/psi)}
\symitem{$P$}{Pressure (psia)}
\symitem{$Z$}{Gas compressibility factor (dimensionless)}
\end{itemize}

\subsection{Gas Pseudopressure ($\psi$)}

\subsubsection{Description}
Gas pseudopressure (or real gas potential) linearizes the gas diffusivity equation and is used in well test analysis.

\begin{tcolorbox}[colback=formulabox,colframe=black!75]
\begin{equation}
\psi(P) = 2 \int_{P_0}^{P} \frac{P'}{\mu_g Z} dP' \quad \text{(psia²/cP)}
\end{equation}
\end{tcolorbox}

\textbf{Symbol Definitions:}
\begin{itemize}[leftmargin=2cm]
\symitem{$\psi$}{Gas pseudopressure (psia²/cP)}
\symitem{$P$}{Pressure (psia)}
\symitem{$P_0$}{Reference pressure (usually 0 psia)}
\symitem{$\mu_g$}{Gas viscosity (cP)}
\symitem{$Z$}{Gas compressibility factor (dimensionless)}
\end{itemize}

\textbf{Note:} Integral is evaluated numerically using trapezoidal rule or Simpson's rule.

\newpage

%=============================================================================
\section{Inflow Performance (Well Productivity)}
%=============================================================================

\subsection{Oil Rate - Radial Flow (Vertical Wells)}

\subsubsection{Description}
Calculates oil production rate for vertical wells with radial flow geometry using Darcy's law for steady-state or pseudo-steady-state flow.

\subsubsection{Darcy's Law - Undersaturated Reservoir}

\begin{tcolorbox}[colback=formulabox,colframe=black!75]
\begin{equation}
q_o = \frac{0.00708 \times k \times h \times (P_i - P_{wf})}{\mu_o \times B_o \times \left[\ln\left(\frac{r_e}{r_w}\right) - 0.75 + S\right]} \quad \text{(STB/day)}
\end{equation}
\end{tcolorbox}

\textbf{Symbol Definitions:}
\begin{itemize}[leftmargin=2cm]
\symitem{$q_o$}{Oil production rate (STB/day)}
\symitem{$k$}{Permeability (mD)}
\symitem{$h$}{Net pay thickness (ft)}
\symitem{$P_i$}{Initial/reservoir pressure (psia)}
\symitem{$P_{wf}$}{Wellbore flowing pressure (psia)}
\symitem{$\mu_o$}{Oil viscosity (cP)}
\symitem{$B_o$}{Oil formation volume factor (rb/stb)}
\symitem{$r_e$}{Drainage radius (ft)}
\symitem{$r_w$}{Wellbore radius (ft)}
\symitem{$S$}{Skin factor (dimensionless)}
\end{itemize}

\subsubsection{Vogel IPR - Saturated Reservoir}

For saturated reservoirs ($P_i < P_b$), use Vogel's correlation:

\begin{tcolorbox}[colback=formulabox,colframe=black!75]
\begin{equation}
\frac{q_o}{q_{o,max}} = 1 - 0.2 \left(\frac{P_{wf}}{P_i}\right) - 0.8 \left(\frac{P_{wf}}{P_i}\right)^2
\end{equation}

where maximum rate is:
\begin{equation}
q_{o,max} = q_o(P_{wf}=0)
\end{equation}
\end{tcolorbox}

\textbf{Additional Symbols:}
\begin{itemize}[leftmargin=2cm]
\symitem{$q_{o,max}$}{Maximum oil rate at zero bottomhole pressure (STB/day)}
\symitem{$P_b$}{Bubble point pressure (psia)}
\end{itemize}

\subsection{Oil Rate - Linear Flow (Horizontal Wells)}

\subsubsection{Description}
Calculates oil production rate for horizontal wells or linear flow geometry.

\begin{tcolorbox}[colback=formulabox,colframe=black!75]
\begin{equation}
q_o = \frac{0.00708 \times k \times h \times L \times (P_i - P_{wf})}{\mu_o \times B_o \times \left[\frac{r_e}{2} + S\right]} \quad \text{(STB/day)}
\end{equation}
\end{tcolorbox}

\textbf{Additional Symbols:}
\begin{itemize}[leftmargin=2cm]
\symitem{$L$}{Horizontal well length (ft)}
\end{itemize}

\subsection{Gas Rate - Radial Flow (Vertical Wells)}

\subsubsection{Description}
Calculates gas production rate for vertical wells using pseudopressure formulation.

\begin{tcolorbox}[colback=formulabox,colframe=black!75]
\begin{equation}
q_g = \frac{0.00708 \times k \times h \times [\psi(P_i) - \psi(P_{wf})]}{\left[\ln\left(\frac{r_e}{r_w}\right) - 0.75 + S\right]} \quad \text{(MSCF/day)}
\end{equation}
\end{tcolorbox}

\textbf{Symbol Definitions:}
\begin{itemize}[leftmargin=2cm]
\symitem{$q_g$}{Gas production rate (MSCF/day)}
\symitem{$\psi(P)$}{Gas pseudopressure (psia²/cP)}
\symitem{$k$}{Permeability (mD)}
\symitem{$h$}{Net pay thickness (ft)}
\symitem{$P_i$}{Reservoir pressure (psia)}
\symitem{$P_{wf}$}{Wellbore flowing pressure (psia)}
\symitem{$r_e$}{Drainage radius (ft)}
\symitem{$r_w$}{Wellbore radius (ft)}
\symitem{$S$}{Skin factor (dimensionless)}
\end{itemize}

\subsection{Gas Rate - Linear Flow (Horizontal Wells)}

\begin{tcolorbox}[colback=formulabox,colframe=black!75]
\begin{equation}
q_g = \frac{0.00708 \times k \times h \times L \times [\psi(P_i) - \psi(P_{wf})]}{\left[\frac{r_e}{2} + S\right]} \quad \text{(MSCF/day)}
\end{equation}
\end{tcolorbox}

\textbf{Additional Symbols:}
\begin{itemize}[leftmargin=2cm]
\symitem{$L$}{Horizontal well length (ft)}
\end{itemize}

\newpage

%=============================================================================
\section{Relative Permeability Correlations}
%=============================================================================

\subsection{Corey (1954) Correlation}

\subsubsection{Description}
Simple power-law model for relative permeability as a function of normalized saturation.

\subsubsection{Normalized Saturation}

\begin{tcolorbox}[colback=formulabox,colframe=black!75]
\textbf{Water Phase:}
\begin{equation}
S_w^* = \frac{S_w - S_{wc}}{1 - S_{wc} - S_{or}}
\end{equation}

\textbf{Oil Phase (Water-Oil):}
\begin{equation}
S_o^* = \frac{S_o - S_{or}}{1 - S_{wc} - S_{or}}
\end{equation}

\textbf{Gas Phase:}
\begin{equation}
S_g^* = \frac{S_g - S_{gc}}{1 - S_{wc} - S_{org}}
\end{equation}
\end{tcolorbox}

\textbf{Symbol Definitions:}
\begin{itemize}[leftmargin=2cm]
\symitem{$S_w^*$}{Normalized water saturation}
\symitem{$S_o^*$}{Normalized oil saturation}
\symitem{$S_g^*$}{Normalized gas saturation}
\symitem{$S_w$}{Water saturation}
\symitem{$S_o$}{Oil saturation}
\symitem{$S_g$}{Gas saturation}
\symitem{$S_{wc}$}{Connate water saturation}
\symitem{$S_{or}$}{Residual oil saturation (to water)}
\symitem{$S_{org}$}{Residual oil saturation (to gas)}
\symitem{$S_{gc}$}{Critical gas saturation}
\end{itemize}

\subsubsection{Relative Permeability Functions}

\begin{tcolorbox}[colback=formulabox,colframe=black!75]
\textbf{Water Relative Permeability:}
\begin{equation}
k_{rw} = k_{rw}^{max} \times (S_w^*)^{n_w}
\end{equation}

\textbf{Oil Relative Permeability (to water):}
\begin{equation}
k_{row} = k_{row}^{max} \times (1 - S_w^*)^{n_o}
\end{equation}

\textbf{Gas Relative Permeability:}
\begin{equation}
k_{rg} = k_{rg}^{max} \times (S_g^*)^{n_g}
\end{equation}

\textbf{Oil Relative Permeability (to gas):}
\begin{equation}
k_{rog} = k_{rog}^{max} \times (1 - S_g^*)^{n_o}
\end{equation}
\end{tcolorbox}

\textbf{Additional Symbols:}
\begin{itemize}[leftmargin=2cm]
\symitem{$k_{rw}$}{Water relative permeability}
\symitem{$k_{row}$}{Oil relative permeability (to water)}
\symitem{$k_{rg}$}{Gas relative permeability}
\symitem{$k_{rog}$}{Oil relative permeability (to gas)}
\symitem{$k_{rw}^{max}$}{Maximum water relative permeability}
\symitem{$k_{row}^{max}$}{Maximum oil relative permeability}
\symitem{$k_{rg}^{max}$}{Maximum gas relative permeability}
\symitem{$n_w$}{Corey exponent for water}
\symitem{$n_o$}{Corey exponent for oil}
\symitem{$n_g$}{Corey exponent for gas}
\end{itemize}

\textbf{Typical Values:}
\begin{itemize}
\item $n_w$: 1.5 - 3.0
\item $n_o$: 2.0 - 4.0
\item $n_g$: 2.0 - 3.5
\end{itemize}

\subsection{LET Correlation (Lomeland-Ebeltoft-Thomas 2005)}

\subsubsection{Description}
Flexible three-parameter model (L, E, T) that can fit complex relative permeability curves.

\begin{tcolorbox}[colback=formulabox,colframe=black!75]
\textbf{Water Relative Permeability:}
\begin{equation}
k_{rw} = k_{rw}^{max} \times \frac{(S_w^*)^{L_w}}{(S_w^*)^{L_w} + E_w (1 - S_w^*)^{T_w}}
\end{equation}

\textbf{Oil Relative Permeability:}
\begin{equation}
k_{ro} = k_{ro}^{max} \times \frac{(1 - S_w^*)^{L_o}}{(1 - S_w^*)^{L_o} + E_o (S_w^*)^{T_o}}
\end{equation}

\textbf{Gas Relative Permeability:}
\begin{equation}
k_{rg} = k_{rg}^{max} \times \frac{(S_g^*)^{L_g}}{(S_g^*)^{L_g} + E_g (1 - S_g^*)^{T_g}}
\end{equation}
\end{tcolorbox}

\textbf{Symbol Definitions:}
\begin{itemize}[leftmargin=2cm]
\symitem{$L_w, L_o, L_g$}{Low saturation parameters (controls curve near $S^*=0$)}
\symitem{$E_w, E_o, E_g$}{Elevation parameters (controls height/position)}
\symitem{$T_w, T_o, T_g$}{Top saturation parameters (controls curve near $S^*=1$)}
\end{itemize}

\textbf{Typical Ranges:}
\begin{itemize}
\item $L$: 1.0 - 3.0
\item $E$: 1.0 - 2.0
\item $T$: 1.0 - 3.0
\end{itemize}

\textbf{Advantages over Corey:}
\begin{itemize}
\item More flexible curve shapes
\item Better fitting to experimental data
\item Independent control of endpoint behavior
\item Smooth transitions
\end{itemize}

\newpage

%=============================================================================
\section{Brine Properties}
%=============================================================================

\textbf{Important Note:} pyResToolbox uses \textbf{different calculation methods} for CH$_4$-saturated versus CO$_2$-saturated brines. The methods are described separately below.

\subsection{CH$_4$-Saturated Brine Properties}

\subsubsection{Description}
For methane-saturated brines, pyResToolbox uses the Modified Spivey Correlation per McCain \textit{Petroleum Reservoir Fluid Properties}, Chapter 4, page 160.

\subsubsection{Calculation Method}

\begin{tcolorbox}[colback=formulabox,colframe=black!75]
\textbf{General Approach:}

Properties are calculated using empirical correlations that account for:
\begin{itemize}
\item Temperature and pressure effects
\item Salinity (weight\% NaCl, range 0-100\%)
\item Methane saturation fraction (0 = no methane, 1 = fully saturated)
\end{itemize}

\textbf{Key Features:}
\begin{itemize}
\item Density calculated from pure water density with salt and methane corrections
\item Viscosity uses temperature-dependent base correlation with salinity adjustments
\item Compressibility derived from density derivatives with respect to pressure
\item Methane solubility from modified Henry's Law with salt correction (lambda and eta coefficients)
\item Gas molar volume calculated using Z-factor correlation for methane
\item Formation volume factor (Bw) accounts for molar volume changes due to dissolved methane
\end{itemize}
\end{tcolorbox}

\textbf{Properties Calculated:}
\begin{itemize}[leftmargin=2cm]
\symitem{$B_w$}{Water formation volume factor (rb/stb)}
\symitem{$\rho_w$}{Brine density (specific gravity, dimensionless)}
\symitem{$\mu_w$}{Brine viscosity (cP)}
\symitem{$c_w$}{Brine compressibility (1/psi)}
\symitem{$R_w$}{Methane solubility (scf/stb)}
\end{itemize}

\textbf{Reference:} McCain, W.D. (2011). \textit{Petroleum Reservoir Fluid Property Correlations}, Chapter 4.

\subsection{CO$_2$-Saturated Brine Properties}

\subsubsection{Description}
For CO$_2$-saturated brines, pyResToolbox uses a thermodynamic approach with multiple specialized correlations that differ significantly from the methane case.

\subsubsection{Calculation Methods}

\begin{tcolorbox}[colback=formulabox,colframe=black!75]
\textbf{Phase Equilibrium:}
\begin{itemize}
\item \textbf{Spycher \& Pruess (2010)}: Modified Redlich-Kwong (SRK) Cubic Equation of State
\item Provides CO$_2$-H$_2$O mutual solubility via activity coefficients
\item Temperature-dependent EOS coefficients (different for T < 99°C vs T > 109°C)
\item Blended calculations for 99-109°C transition range
\end{itemize}

\textbf{Pure Brine Density:}
\begin{itemize}
\item \textbf{Spivey et al.} correlation (modified)
\item Reference: McCain \textit{Petroleum Reservoir Fluid Property Correlations}, Chapter 4
\end{itemize}

\textbf{CO$_2$-Corrected Brine Density:}
\begin{itemize}
\item \textbf{Garcia (2001)} equation for CO$_2$ molar volume
\item Combined with Spycher \& Pruess $x_{CO_2}$ and Spivey base density
\end{itemize}

\textbf{Pure Brine Viscosity:}
\begin{itemize}
\item \textbf{Mao-Duan (2009)} correlation
\end{itemize}

\textbf{CO$_2$-Corrected Brine Viscosity:}
\begin{itemize}
\item \textbf{Islam-Carlson (2012)} model
\item Adjusts base viscosity for dissolved CO$_2$ mole fraction
\end{itemize}
\end{tcolorbox}

\textbf{Properties Calculated:}
\begin{itemize}[leftmargin=2cm]
\symitem{$x_{CO_2}$}{CO$_2$ mole fraction in brine}
\symitem{$y_{H_2O}$}{H$_2$O mole fraction in CO$_2$-rich phase}
\symitem{$\rho_{CO_2-gas}$}{CO$_2$-rich gas density (lb/ft³)}
\symitem{$\rho_{brine-sat}$}{CO$_2$-saturated brine density (lb/ft³)}
\symitem{$\rho_{brine-pure}$}{Pure brine density (lb/ft³)}
\symitem{$\mu_{brine-sat}$}{CO$_2$-saturated brine viscosity (cP)}
\symitem{$\mu_{brine-pure}$}{Pure brine viscosity (cP)}
\symitem{$B_{w-sat}$}{CO$_2$-saturated brine FVF (rb/stb)}
\symitem{$R_w$}{CO$_2$ solubility (scf/stb)}
\symitem{$c_{w-unsat}$}{Undersaturated brine compressibility (1/psi)}
\symitem{$c_{w-sat}$}{Saturated brine compressibility (1/psi)}
\end{itemize}

\textbf{Salinity Input:} Parts per million (ppm) NaCl

\textbf{Key References:}
\begin{itemize}
\item Spycher, N. and Pruess, K. (2010). Geochimica et Cosmochimica Acta
\item Garcia, J.E. (2001). Fluid Phase Equilibria
\item Mao, S. and Duan, Z. (2009). Chemical Geology
\item Islam, A.W. and Carlson, E.S. (2012). Geothermics
\end{itemize}

\subsection{Key Differences Between CH$_4$ and CO$_2$ Methods}

\begin{table}[h]
\centering
\begin{tabular}{p{3cm}p{5cm}p{5cm}}
\toprule
\textbf{Aspect} & \textbf{CH$_4$-Saturated} & \textbf{CO$_2$-Saturated} \\
\midrule
Phase Equilibrium & Henry's Law-based empirical & Cubic EOS (Spycher \& Pruess) \\
Density Correction & Dissolved CH$_4$ molar volume & Garcia (2001) CO$_2$ molar volume \\
Viscosity Adjustment & Salinity effect factors & Islam-Carlson (2012) model \\
Salinity Units & Weight\% NaCl (0-100) & ppm NaCl \\
Compressibility & From density derivatives & Numerical differentiation (sat/unsat) \\
\bottomrule
\end{tabular}
\caption{Comparison of CH$_4$ vs CO$_2$ Brine Calculation Methods}
\end{table}

\newpage

%=============================================================================
\section{Reservoir Heterogeneity}
%=============================================================================

\subsection{Lorenz Coefficient}

\subsubsection{Description}
The Lorenz coefficient quantifies reservoir heterogeneity based on the cumulative flow capacity vs cumulative storage capacity curve. It ranges from 0 (homogeneous) to 1 (completely heterogeneous).

\begin{tcolorbox}[colback=formulabox,colframe=black!75]
\begin{equation}
L = 2 \times \int_0^1 [F(\phi) - \phi] d\phi
\end{equation}

where $F(\phi)$ is the Lorenz curve:
\begin{equation}
F(\phi) = \frac{\text{Cumulative Flow Capacity}}{\text{Total Flow Capacity}}
\end{equation}
\end{tcolorbox}

\textbf{Symbol Definitions:}
\begin{itemize}[leftmargin=2cm]
\symitem{$L$}{Lorenz coefficient (0-1)}
\symitem{$F(\phi)$}{Lorenz curve function}
\symitem{$\phi$}{Cumulative storage fraction}
\end{itemize}

\textbf{Interpretation:}
\begin{itemize}
\item $L = 0$: Perfectly homogeneous (all layers have same $k h$)
\item $L = 0.3$: Low heterogeneity
\item $L = 0.5$: Moderate heterogeneity
\item $L = 0.7$: High heterogeneity
\item $L \rightarrow 1$: Extreme heterogeneity
\end{itemize}

\subsection{Dykstra-Parsons Coefficient ($\beta$)}

\subsubsection{Description}
The Dykstra-Parsons coefficient measures permeability variation assuming a log-normal distribution.

\begin{tcolorbox}[colback=formulabox,colframe=black!75]
\begin{equation}
\beta = \frac{k_{50} - k_{84.1}}{k_{50}}
\end{equation}

where:
\begin{itemize}
\item $k_{50}$ is the median permeability (50th percentile)
\item $k_{84.1}$ is the 84.1th percentile permeability (1 standard deviation below mean in log space)
\end{itemize}
\end{tcolorbox}

\textbf{Symbol Definitions:}
\begin{itemize}[leftmargin=2cm]
\symitem{$\beta$}{Dykstra-Parsons coefficient (0-1)}
\symitem{$k_{50}$}{Median permeability (mD)}
\symitem{$k_{84.1}$}{84.1th percentile permeability (mD)}
\end{itemize}

\textbf{Interpretation:}
\begin{itemize}
\item $\beta < 0.5$: Low variation (homogeneous)
\item $\beta = 0.5 - 0.7$: Moderate variation
\item $\beta > 0.7$: High variation (heterogeneous)
\end{itemize}

\subsection{Lorenz-Beta Conversion}

\subsubsection{Description}
Conversion between Lorenz coefficient and Dykstra-Parsons beta parameter, assuming log-normal permeability distribution.

\begin{tcolorbox}[colback=formulabox,colframe=black!75]
\textbf{Beta to Lorenz:}
\begin{equation}
L \approx 1 - \frac{1}{\sqrt{2\pi}} \int_{-\infty}^{\infty} \exp\left(-\frac{x^2}{2}\right) \times \exp(x \sigma) dx
\end{equation}

where $\sigma$ is related to $\beta$ through:
\begin{equation}
\beta = 1 - \exp(-\sigma)
\end{equation}

\textbf{Approximate Conversion:}
\begin{equation}
L \approx 0.632 \times \beta^{0.9}
\end{equation}
\end{tcolorbox}

\textbf{Symbol Definitions:}
\begin{itemize}[leftmargin=2cm]
\symitem{$L$}{Lorenz coefficient}
\symitem{$\beta$}{Dykstra-Parsons coefficient}
\symitem{$\sigma$}{Standard deviation of $\ln(k)$}
\end{itemize}

\subsection{Layer Permeability Distribution}

\subsubsection{Description}
Generates permeability distribution for layered reservoirs based on Lorenz coefficient.

\begin{tcolorbox}[colback=formulabox,colframe=black!75]
\textbf{Log-Normal Distribution:}
\begin{equation}
k_i = k_{avg} \times \exp\left[\sigma \times \Phi^{-1}(p_i) - \frac{\sigma^2}{2}\right]
\end{equation}

where:
\begin{itemize}
\item $k_i$ is permeability of layer $i$
\item $k_{avg}$ is average permeability
\item $\Phi^{-1}$ is inverse standard normal CDF
\item $p_i = (i - 0.5)/N$ is cumulative probability
\item $N$ is number of layers
\end{itemize}
\end{tcolorbox}

\textbf{Symbol Definitions:}
\begin{itemize}[leftmargin=2cm]
\symitem{$k_i$}{Layer permeability (mD)}
\symitem{$k_{avg}$}{Average permeability (mD)}
\symitem{$\sigma$}{Standard deviation of $\ln(k)$}
\symitem{$\Phi^{-1}$}{Inverse cumulative normal distribution}
\symitem{$N$}{Number of layers}
\end{itemize}

\newpage

%=============================================================================
\section{Aquifer Models}
%=============================================================================

\subsection{Van Everdingen-Hurst Model}

\subsubsection{Description}
The Van Everdingen-Hurst model calculates water influx from an aquifer into a reservoir. It uses dimensionless influence functions based on the radial diffusivity equation.

\begin{tcolorbox}[colback=formulabox,colframe=black!75]
\textbf{Water Influx:}
\begin{equation}
W_e(t) = B \times \sum_{j=1}^{n} \Delta P_j \times W_D(t_D - t_{D,j})
\end{equation}

where:
\begin{equation}
B = 1.119 \times \phi \times c_t \times h \times r_e^2 \quad \text{(rb/psi)}
\end{equation}

\textbf{Dimensionless Time:}
\begin{equation}
t_D = \frac{0.000264 \times k \times t}{\phi \times \mu \times c_t \times r_e^2}
\end{equation}

\textbf{Dimensionless Radius:}
\begin{equation}
r_D = \frac{r_{aq}}{r_e}
\end{equation}
\end{tcolorbox}

\textbf{Symbol Definitions:}
\begin{itemize}[leftmargin=2cm]
\symitem{$W_e$}{Cumulative water influx (rb)}
\symitem{$B$}{Aquifer constant (rb/psi)}
\symitem{$\Delta P_j$}{Pressure drop at time step $j$ (psia)}
\symitem{$W_D$}{Dimensionless water influx function}
\symitem{$t_D$}{Dimensionless time}
\symitem{$\phi$}{Aquifer porosity (fraction)}
\symitem{$c_t$}{Total compressibility (1/psi)}
\symitem{$h$}{Aquifer thickness (ft)}
\symitem{$r_e$}{Reservoir radius (ft)}
\symitem{$r_{aq}$}{Aquifer outer radius (ft)}
\symitem{$k$}{Aquifer permeability (mD)}
\symitem{$\mu$}{Water viscosity (cP)}
\symitem{$t$}{Time (days)}
\end{itemize}

\subsection{Influence Functions}

The dimensionless water influx $W_D(t_D)$ is tabulated for various dimensionless radius ratios $r_D$. For:
\begin{itemize}
\item Infinite aquifer: $r_D = \infty$
\item Finite aquifer: $r_D = 2, 3, 5, 10, etc.$
\end{itemize}

\newpage

%=============================================================================
\section{Geomechanics and Wellbore Stability}
%=============================================================================

The geomechanics module provides 27 tools for stress analysis, wellbore stability, and reservoir geomechanics calculations. These are critical for drilling optimization, completion design, sand management, and induced seismicity assessment.

\subsection{Vertical (Overburden) Stress}

\subsubsection{Description}
The vertical stress at any depth is the weight of overlying rock and fluid column.

\begin{tcolorbox}[colback=formulabox,colframe=black!75]
\textbf{Onshore Wells:}
\begin{equation}
\sigma_v = \frac{\bar{\rho}_{bulk} \times D}{144} \quad \text{(psia)}
\end{equation}

\textbf{Offshore Wells:}
\begin{equation}
\sigma_v = \frac{\rho_w \times D_w + \bar{\rho}_{bulk} \times (D - D_w)}{144} \quad \text{(psia)}
\end{equation}

\textbf{Overburden Gradient:}
\begin{equation}
\frac{d\sigma_v}{dD} = \frac{\bar{\rho}_{bulk}}{144} \quad \text{(psi/ft)}
\end{equation}
\end{tcolorbox}

\textbf{Symbol Definitions:}
\begin{itemize}[leftmargin=2cm]
\symitem{$\sigma_v$}{Vertical stress (psia)}
\symitem{$\bar{\rho}_{bulk}$}{Average bulk density of overburden (lb/ft$^3$)}
\symitem{$D$}{True vertical depth below surface (ft)}
\symitem{$D_w$}{Water depth for offshore wells (ft)}
\symitem{$\rho_w$}{Water density (lb/ft$^3$) - 62.4 fresh, 64 seawater}
\end{itemize}

\textbf{Typical Values:}
\begin{itemize}
\item Average bulk density: 140-160 lb/ft$^3$
\item Overburden gradient: 0.95-1.1 psi/ft
\end{itemize}

\subsection{Pore Pressure Prediction}

\subsubsection{Eaton Method (1975)}

\begin{tcolorbox}[colback=formulabox,colframe=black!75]
\textbf{Sonic Method:}
\begin{equation}
P_p = \sigma_v - (\sigma_v - P_{p,n}) \times \left(\frac{\Delta t_n}{\Delta t}\right)^{n}
\end{equation}

\textbf{Resistivity Method:}
\begin{equation}
P_p = \sigma_v - (\sigma_v - P_{p,n}) \times \left(\frac{R}{R_n}\right)^{n}
\end{equation}
\end{tcolorbox}

\textbf{Symbol Definitions:}
\begin{itemize}[leftmargin=2cm]
\symitem{$P_p$}{Pore pressure (psia)}
\symitem{$P_{p,n}$}{Normal pore pressure at depth (psia)}
\symitem{$\Delta t$}{Observed sonic transit time ($\mu$s/ft)}
\symitem{$\Delta t_n$}{Normal compaction trend sonic value ($\mu$s/ft)}
\symitem{$R$}{Observed resistivity (ohm-m)}
\symitem{$R_n$}{Normal compaction trend resistivity (ohm-m)}
\symitem{$n$}{Eaton exponent (3.0 for sonic, 1.2 for resistivity)}
\end{itemize}

\subsection{Effective Stress}

\subsubsection{Terzaghi and Biot Effective Stress}

\begin{tcolorbox}[colback=formulabox,colframe=black!75]
\textbf{Terzaghi Effective Stress ($\alpha = 1$):}
\begin{equation}
\sigma' = \sigma - P_p
\end{equation}

\textbf{Biot Effective Stress:}
\begin{equation}
\sigma' = \sigma - \alpha P_p
\end{equation}

where Biot coefficient:
\begin{equation}
\alpha = 1 - \frac{K_{bulk}}{K_{grain}} \approx 0.6 \text{ to } 1.0
\end{equation}
\end{tcolorbox}

\textbf{Symbol Definitions:}
\begin{itemize}[leftmargin=2cm]
\symitem{$\sigma'$}{Effective stress (psi)}
\symitem{$\sigma$}{Total stress (psi)}
\symitem{$P_p$}{Pore pressure (psi)}
\symitem{$\alpha$}{Biot coefficient (dimensionless)}
\symitem{$K_{bulk}$}{Rock bulk modulus (psi)}
\symitem{$K_{grain}$}{Grain (matrix) modulus (psi)}
\end{itemize}

\subsection{Horizontal Stress}

\subsubsection{Poroelastic Horizontal Stress Model}

\begin{tcolorbox}[colback=formulabox,colframe=black!75]
\textbf{Minimum Horizontal Stress (Uniaxial Strain):}
\begin{equation}
\sigma_h = \frac{\nu}{1-\nu}(\sigma_v - \alpha P_p) + \alpha P_p
\end{equation}

\textbf{With Tectonic Stress:}
\begin{equation}
\sigma_h = \frac{\nu}{1-\nu}(\sigma_v - \alpha P_p) + \alpha P_p + \sigma_{tect}
\end{equation}

\textbf{Stress Regimes:}
\begin{itemize}
\item Normal Faulting: $\sigma_v > \sigma_H > \sigma_h$
\item Strike-Slip: $\sigma_H > \sigma_v > \sigma_h$
\item Reverse Faulting: $\sigma_H > \sigma_h > \sigma_v$
\end{itemize}
\end{tcolorbox}

\textbf{Symbol Definitions:}
\begin{itemize}[leftmargin=2cm]
\symitem{$\sigma_h$}{Minimum horizontal stress (psi)}
\symitem{$\sigma_H$}{Maximum horizontal stress (psi)}
\symitem{$\nu$}{Poisson's ratio (0.15-0.40 typical)}
\symitem{$\sigma_{tect}$}{Tectonic stress contribution (psi)}
\end{itemize}

\subsection{Elastic Moduli Relationships}

\subsubsection{Conversion Between Elastic Constants}

\begin{tcolorbox}[colback=formulabox,colframe=black!75]
\textbf{Given Young's Modulus (E) and Poisson's Ratio ($\nu$):}
\begin{align}
K &= \frac{E}{3(1-2\nu)} \quad \text{(Bulk Modulus)} \\
G &= \frac{E}{2(1+\nu)} \quad \text{(Shear Modulus)} \\
\lambda &= \frac{E\nu}{(1+\nu)(1-2\nu)} \quad \text{(Lamé's First Parameter)}
\end{align}

\textbf{Plane Strain Modulus:}
\begin{equation}
E' = \frac{E}{1-\nu^2}
\end{equation}
\end{tcolorbox}

\textbf{Symbol Definitions:}
\begin{itemize}[leftmargin=2cm]
\symitem{$E$}{Young's modulus (psi)}
\symitem{$K$}{Bulk modulus (psi)}
\symitem{$G$}{Shear modulus (psi)}
\symitem{$\lambda$}{Lamé's first parameter (psi)}
\symitem{$\nu$}{Poisson's ratio (dimensionless)}
\symitem{$E'$}{Plane strain modulus (psi)}
\end{itemize}

\subsection{Rock Strength}

\subsubsection{Mohr-Coulomb Failure Criterion}

\begin{tcolorbox}[colback=formulabox,colframe=black!75]
\textbf{Shear Failure Envelope:}
\begin{equation}
\tau = c + \sigma_n \tan\phi
\end{equation}

\textbf{Principal Stress Form:}
\begin{equation}
\sigma_1 = \text{UCS} + q \sigma_3
\end{equation}

where:
\begin{align}
q &= \frac{1 + \sin\phi}{1 - \sin\phi} = \tan^2\left(45° + \frac{\phi}{2}\right) \\
\text{UCS} &= \frac{2c \cos\phi}{1 - \sin\phi}
\end{align}
\end{tcolorbox}

\textbf{Symbol Definitions:}
\begin{itemize}[leftmargin=2cm]
\symitem{$\tau$}{Shear stress at failure (psi)}
\symitem{$c$}{Cohesion (psi)}
\symitem{$\sigma_n$}{Normal stress on failure plane (psi)}
\symitem{$\phi$}{Internal friction angle (degrees)}
\symitem{$\sigma_1$}{Maximum principal stress at failure (psi)}
\symitem{$\sigma_3$}{Minimum principal stress (confining stress) (psi)}
\symitem{$q$}{Mohr-Coulomb strength parameter}
\symitem{UCS}{Unconfined Compressive Strength (psi)}
\end{itemize}

\subsection{Fracture Gradient}

\subsubsection{Estimation Methods}

\begin{tcolorbox}[colback=formulabox,colframe=black!75]
\textbf{Eaton Method (1969):}
\begin{equation}
\frac{P_{frac}}{D} = \frac{\nu}{1-\nu}\left(\frac{\sigma_v - P_p}{D}\right) + \frac{P_p}{D}
\end{equation}

\textbf{From Known $\sigma_h$:}
\begin{equation}
P_{frac} = \sigma_h \quad \text{(minimum case)}
\end{equation}

\textbf{Equivalent Mud Weight:}
\begin{equation}
\text{EMW}_{frac} = \frac{P_{frac}}{0.052 \times D} \quad \text{(ppg)}
\end{equation}
\end{tcolorbox}

\textbf{Symbol Definitions:}
\begin{itemize}[leftmargin=2cm]
\symitem{$P_{frac}$}{Fracture pressure (psi)}
\symitem{EMW$_{frac}$}{Fracture equivalent mud weight (ppg)}
\symitem{$D$}{True vertical depth (ft)}
\end{itemize}

\subsection{Wellbore Stress (Kirsch Equations)}

\subsubsection{Stress at Wellbore Wall}

\begin{tcolorbox}[colback=formulabox,colframe=black!75]
\textbf{Hoop (Tangential) Stress at Wall:}
\begin{equation}
\sigma_\theta = \sigma_H + \sigma_h - 2(\sigma_H - \sigma_h)\cos 2\theta - P_w
\end{equation}

\textbf{Maximum Hoop Stress (at $\theta = 90°$):}
\begin{equation}
\sigma_{\theta,max} = 3\sigma_H - \sigma_h - P_w
\end{equation}

\textbf{Minimum Hoop Stress (at $\theta = 0°$):}
\begin{equation}
\sigma_{\theta,min} = 3\sigma_h - \sigma_H - P_w
\end{equation}

\textbf{Radial Stress at Wall:}
\begin{equation}
\sigma_r = P_w
\end{equation}
\end{tcolorbox}

\textbf{Symbol Definitions:}
\begin{itemize}[leftmargin=2cm]
\symitem{$\sigma_\theta$}{Tangential (hoop) stress (psi)}
\symitem{$\sigma_r$}{Radial stress (psi)}
\symitem{$P_w$}{Wellbore pressure (mud pressure) (psi)}
\symitem{$\theta$}{Angle from $\sigma_H$ direction (degrees)}
\end{itemize}

\subsection{Breakout Width Calculation}

\subsubsection{Description}
Borehole breakouts occur when hoop stress exceeds rock strength. The angular width indicates stress state.

\begin{tcolorbox}[colback=formulabox,colframe=black!75]
\textbf{Breakout Initiation Condition:}
\begin{equation}
\sigma_\theta(\theta_b) = \text{UCS} + q \times \sigma_r
\end{equation}

\textbf{Breakout Width:}
\begin{equation}
W_{bo} = \pi - 2\theta_b \quad \text{(radians)}
\end{equation}

where $\theta_b$ is angle from $\sigma_H$ where failure initiates.
\end{tcolorbox}

\subsection{Tensile Failure and Breakdown Pressure}

\subsubsection{Hydraulic Fracturing Initiation}

\begin{tcolorbox}[colback=formulabox,colframe=black!75]
\textbf{Breakdown Pressure (Impermeable Rock):}
\begin{equation}
P_{bd} = 3\sigma_h - \sigma_H + T_0 - P_p
\end{equation}

\textbf{Breakdown Pressure (Permeable Rock):}
\begin{equation}
P_{bd} = \frac{3\sigma_h - \sigma_H + T_0 - \eta P_p}{1 - \eta}
\end{equation}

where poroelastic constant:
\begin{equation}
\eta = \frac{\alpha(1-2\nu)}{1-\nu}
\end{equation}

\textbf{Fracture Re-opening Pressure:}
\begin{equation}
P_{reopen} = 3\sigma_h - \sigma_H - P_p
\end{equation}

\textbf{Propagation Pressure:}
\begin{equation}
P_{prop} \approx \sigma_h
\end{equation}
\end{tcolorbox}

\textbf{Symbol Definitions:}
\begin{itemize}[leftmargin=2cm]
\symitem{$P_{bd}$}{Formation breakdown pressure (psi)}
\symitem{$T_0$}{Tensile strength of rock (psi)}
\symitem{$\eta$}{Poroelastic constant (dimensionless)}
\symitem{$P_{reopen}$}{Fracture re-opening pressure (psi)}
\symitem{$P_{prop}$}{Fracture propagation pressure (psi)}
\end{itemize}

\subsection{Safe Mud Weight Window}

\begin{tcolorbox}[colback=formulabox,colframe=black!75]
\textbf{Lower Bound (Kick Pressure):}
\begin{equation}
\text{MW}_{min} = \frac{P_p}{0.052 \times D} + \Delta_{OB} \quad \text{(ppg)}
\end{equation}

\textbf{Upper Bound (Lost Circulation):}
\begin{equation}
\text{MW}_{max} = \frac{P_{frac}}{0.052 \times D} - \Delta_{frac} \quad \text{(ppg)}
\end{equation}

\textbf{Collapse Bound:}
\begin{equation}
\text{MW}_{collapse} = \frac{P_{collapse}}{0.052 \times D} \quad \text{(ppg)}
\end{equation}
\end{tcolorbox}

\textbf{Symbol Definitions:}
\begin{itemize}[leftmargin=2cm]
\symitem{MW}{Mud weight (ppg)}
\symitem{$\Delta_{OB}$}{Overbalance safety margin (ppg)}
\symitem{$\Delta_{frac}$}{Fracture safety margin (ppg)}
\symitem{$P_{collapse}$}{Collapse pressure from rock mechanics (psi)}
\end{itemize}

\subsection{Reservoir Compaction}

\subsubsection{Uniaxial Strain Compaction}

\begin{tcolorbox}[colback=formulabox,colframe=black!75]
\textbf{Vertical Compaction:}
\begin{equation}
\Delta h = C_m \times \Delta P_p \times h
\end{equation}

\textbf{Uniaxial Compaction Coefficient:}
\begin{equation}
C_m = \frac{\alpha (1+\nu)(1-2\nu)}{E(1-\nu)}
\end{equation}

\textbf{Simplified Form:}
\begin{equation}
\Delta h = \frac{\alpha \Delta P_p \times h}{M}
\end{equation}

where constrained modulus:
\begin{equation}
M = \frac{E(1-\nu)}{(1+\nu)(1-2\nu)}
\end{equation}
\end{tcolorbox}

\textbf{Symbol Definitions:}
\begin{itemize}[leftmargin=2cm]
\symitem{$\Delta h$}{Compaction (ft)}
\symitem{$h$}{Reservoir thickness (ft)}
\symitem{$\Delta P_p$}{Pressure depletion (psi)}
\symitem{$C_m$}{Uniaxial compaction coefficient (1/psi)}
\symitem{$M$}{Constrained (P-wave) modulus (psi)}
\end{itemize}

\subsection{Stress Path During Depletion}

\subsubsection{Horizontal Stress Change with Pore Pressure}

\begin{tcolorbox}[colback=formulabox,colframe=black!75]
\textbf{Stress Path Coefficient:}
\begin{equation}
\gamma = \frac{\Delta\sigma_h}{\Delta P_p} = \alpha \frac{1-2\nu}{1-\nu}
\end{equation}

\textbf{Final Horizontal Stress:}
\begin{equation}
\sigma_h^{final} = \sigma_h^{initial} + \gamma \times \Delta P_p
\end{equation}

\textbf{Effective Stress Change:}
\begin{equation}
\Delta\sigma'_h = \Delta\sigma_h - \alpha \Delta P_p = (\gamma - \alpha)\Delta P_p
\end{equation}
\end{tcolorbox}

\textbf{Symbol Definitions:}
\begin{itemize}[leftmargin=2cm]
\symitem{$\gamma$}{Stress path coefficient (typically 0.5-0.8)}
\symitem{$\Delta\sigma_h$}{Change in horizontal stress (psi)}
\symitem{$\Delta P_p$}{Change in pore pressure (psi)}
\end{itemize}

\textbf{Note:} For depletion ($\Delta P_p < 0$), horizontal stress decreases but effective stress increases.

\subsection{Stress Polygon}

\subsubsection{Frictional Equilibrium Constraints}

\begin{tcolorbox}[colback=formulabox,colframe=black!75]
\textbf{Maximum Stress Ratio (Frictional Limit):}
\begin{equation}
\frac{\sigma_1'}{\sigma_3'} \leq \left(\sqrt{\mu^2 + 1} + \mu\right)^2 = q_f
\end{equation}

\textbf{Normal Faulting Constraint ($\sigma_v = \sigma_1$):}
\begin{equation}
\sigma_h \geq \frac{\sigma_v - P_p}{q_f} + P_p
\end{equation}

\textbf{Reverse Faulting Constraint ($\sigma_v = \sigma_3$):}
\begin{equation}
\sigma_H \leq (\sigma_v - P_p) \times q_f + P_p
\end{equation}
\end{tcolorbox}

\textbf{Symbol Definitions:}
\begin{itemize}[leftmargin=2cm]
\symitem{$\mu$}{Fault friction coefficient (0.6-0.85 typical, Byerlee's law)}
\symitem{$q_f$}{Frictional limit ratio}
\end{itemize}

\subsection{Fault Stability Analysis}

\subsubsection{Coulomb Failure Stress}

\begin{tcolorbox}[colback=formulabox,colframe=black!75]
\textbf{Slip Tendency:}
\begin{equation}
T_s = \frac{\tau}{\sigma_n'}
\end{equation}

\textbf{Dilation Tendency:}
\begin{equation}
T_d = \frac{\sigma_1' - \sigma_n'}{\sigma_1' - \sigma_3'}
\end{equation}

\textbf{Coulomb Failure Stress:}
\begin{equation}
\text{CFS} = \tau - \mu\sigma_n' - c
\end{equation}

\textbf{Slip Condition:}
\begin{equation}
\text{CFS} \geq 0 \quad \text{or} \quad T_s \geq \mu
\end{equation}

\textbf{Critical Pore Pressure for Slip:}
\begin{equation}
P_{p,crit} = \sigma_n - \frac{\tau - c}{\mu}
\end{equation}
\end{tcolorbox}

\textbf{Symbol Definitions:}
\begin{itemize}[leftmargin=2cm]
\symitem{$T_s$}{Slip tendency (dimensionless)}
\symitem{$T_d$}{Dilation tendency (dimensionless)}
\symitem{CFS}{Coulomb Failure Stress (psi)}
\symitem{$\tau$}{Shear stress on fault plane (psi)}
\symitem{$\sigma_n'$}{Effective normal stress on fault (psi)}
\symitem{$c$}{Fault cohesion (psi) - usually 0 for reactivation}
\end{itemize}

\subsection{Thermal Stress Effects}

\begin{tcolorbox}[colback=formulabox,colframe=black!75]
\textbf{Thermal Stress (Plane Strain):}
\begin{equation}
\sigma_{thermal} = -\frac{E \alpha_T \Delta T}{1-\nu}
\end{equation}

\textbf{Effect on Hoop Stress:}
\begin{equation}
\Delta\sigma_\theta = \sigma_{thermal}
\end{equation}

\textbf{Implications:}
\begin{itemize}
\item Cooling ($\Delta T < 0$): Tensile stress, reduces collapse risk, promotes fracturing
\item Heating ($\Delta T > 0$): Compressive stress, increases collapse risk, inhibits fracturing
\end{itemize}
\end{tcolorbox}

\textbf{Symbol Definitions:}
\begin{itemize}[leftmargin=2cm]
\symitem{$\sigma_{thermal}$}{Thermal stress (psi)}
\symitem{$\alpha_T$}{Linear thermal expansion coefficient (1/°F)}
\symitem{$\Delta T$}{Temperature change (°F)}
\end{itemize}

\textbf{Typical Values:}
\begin{itemize}
\item $\alpha_T$: $5 \times 10^{-6}$ to $8 \times 10^{-6}$ 1/°F for rocks
\end{itemize}

\subsection{UCS Estimation from Logs}

\subsubsection{Empirical Correlations}

\begin{tcolorbox}[colback=formulabox,colframe=black!75]
\textbf{McNally (1987) - Sandstone:}
\begin{equation}
\text{UCS} = 1200 \times \exp(-0.036 \times \Delta t) \quad \text{(psi)}
\end{equation}

\textbf{Horsrud (2001) - Shale:}
\begin{equation}
\text{UCS} = 0.77 \times V_p^{2.93} \quad \text{(MPa, $V_p$ in km/s)}
\end{equation}

\textbf{Chang (2006) - General:}
\begin{equation}
\text{UCS} = 2.28 + 4.1089 \times E \quad \text{(MPa, $E$ in GPa)}
\end{equation}

\textbf{Vernik (1993) - Carbonate:}
\begin{equation}
\text{UCS} = 254 \times (1 - 2.7\phi)^2 \quad \text{(MPa)}
\end{equation}
\end{tcolorbox}

\textbf{Symbol Definitions:}
\begin{itemize}[leftmargin=2cm]
\symitem{$\Delta t$}{Sonic transit time ($\mu$s/ft)}
\symitem{$V_p$}{P-wave velocity (km/s)}
\symitem{$\phi$}{Porosity (fraction)}
\end{itemize}

\subsection{Critical Drawdown for Sand Production}

\begin{tcolorbox}[colback=formulabox,colframe=black!75]
\textbf{Critical Flowing BHP:}
\begin{equation}
P_{wf,crit} = \frac{3\sigma_H' - \sigma_h' + (1+q)P_p - \text{UCS}}{1+q}
\end{equation}

\textbf{Critical Drawdown:}
\begin{equation}
\Delta P_{crit} = P_{res} - P_{wf,crit}
\end{equation}

where $q = (1+\sin\phi)/(1-\sin\phi)$
\end{tcolorbox}

\textbf{Risk Assessment:}
\begin{itemize}
\item Low risk: $\Delta P_{crit} > 1000$ psi, UCS $> 2000$ psi
\item Moderate risk: $\Delta P_{crit}$ 500-1000 psi
\item High risk: $\Delta P_{crit} < 500$ psi, UCS $< 1000$ psi
\end{itemize}

\subsection{Multiple Failure Criteria Comparison}

\subsubsection{Available Criteria}

\begin{tcolorbox}[colback=formulabox,colframe=black!75]
\textbf{1. Mohr-Coulomb (ignores $\sigma_2$):}
\begin{equation}
\sigma_1 = \text{UCS} + q\sigma_3
\end{equation}

\textbf{2. Drucker-Prager (smooth cone):}
\begin{equation}
\sqrt{J_2} = k + \alpha I_1
\end{equation}

where $I_1 = \sigma_1 + \sigma_2 + \sigma_3$ and $J_2 = \frac{1}{6}[(\sigma_1-\sigma_2)^2 + (\sigma_2-\sigma_3)^2 + (\sigma_1-\sigma_3)^2]$

\textbf{3. Mogi-Coulomb (includes $\sigma_2$ effect):}
\begin{equation}
\tau_{oct} = a + b \sigma_{m,2}
\end{equation}

where $\sigma_{m,2} = (\sigma_1 + \sigma_3)/2$ and $\tau_{oct} = \frac{\sqrt{2}}{3}\sqrt{(\sigma_1-\sigma_2)^2 + (\sigma_2-\sigma_3)^2 + (\sigma_1-\sigma_3)^2}$
\end{tcolorbox}

\textbf{Note:} Mohr-Coulomb is most conservative. True triaxial criteria (Mogi-Coulomb, Modified Lade) account for intermediate stress strengthening effect.

\newpage

%=============================================================================
\section{Phase Behavior}
%=============================================================================

\subsection{Rachford-Rice Flash Calculation}

\subsubsection{Description}
The Rachford-Rice equation is used to calculate vapor and liquid phase compositions and amounts for a multi-component mixture at specified temperature and pressure.

\begin{tcolorbox}[colback=formulabox,colframe=black!75]
\textbf{Rachford-Rice Equation:}
\begin{equation}
\sum_{i=1}^{n} \frac{z_i (K_i - 1)}{1 + V(K_i - 1)} = 0
\end{equation}

where:
\begin{itemize}
\item $V$ is vapor mole fraction (solve iteratively)
\item $L = 1 - V$ is liquid mole fraction
\end{itemize}

\textbf{Component Mole Fractions:}
\begin{align}
x_i &= \frac{z_i}{1 + V(K_i - 1)} \quad \text{(liquid)} \\
y_i &= \frac{K_i \times z_i}{1 + V(K_i - 1)} \quad \text{(vapor)}
\end{align}
\end{tcolorbox}

\textbf{Symbol Definitions:}
\begin{itemize}[leftmargin=2cm]
\symitem{$z_i$}{Overall mole fraction of component $i$}
\symitem{$x_i$}{Liquid phase mole fraction of component $i$}
\symitem{$y_i$}{Vapor phase mole fraction of component $i$}
\symitem{$K_i$}{Equilibrium K-value for component $i$ ($K_i = y_i/x_i$)}
\symitem{$V$}{Vapor phase mole fraction}
\symitem{$L$}{Liquid phase mole fraction}
\symitem{$n$}{Number of components}
\end{itemize}

\textbf{Solution Method:}
Newton-Raphson iteration:
\begin{equation}
V^{n+1} = V^n - \frac{f(V^n)}{f'(V^n)}
\end{equation}

where:
\begin{align}
f(V) &= \sum_{i=1}^{n} \frac{z_i (K_i - 1)}{1 + V(K_i - 1)} \\
f'(V) &= -\sum_{i=1}^{n} \frac{z_i (K_i - 1)^2}{[1 + V(K_i - 1)]^2}
\end{align}

\newpage

%=============================================================================
\section{Quick Reference Tables}
%=============================================================================

\subsection{Physical Constants}

\begin{table}[h]
\centering
\begin{tabular}{lll}
\toprule
\textbf{Constant} & \textbf{Symbol} & \textbf{Value} \\
\midrule
Universal gas constant & $R$ & 10.73 psia·ft³/(lbmol·°R) \\
Standard pressure & $P_{sc}$ & 14.7 psia \\
Standard temperature & $T_{sc}$ & 60°F (520°R) \\
Air molecular weight & $M_{air}$ & 28.97 lb/lbmol \\
Water density at SC & $\rho_w^{SC}$ & 62.4 lb/ft³ \\
Conversion °F to °R & $T_R$ & $T_F + 459.67$ \\
\bottomrule
\end{tabular}
\caption{Physical Constants Used in Calculations}
\end{table}

\subsection{Conversion Factors}

\begin{table}[h]
\centering
\begin{tabular}{lll}
\toprule
\textbf{From} & \textbf{To} & \textbf{Multiply By} \\
\midrule
psia & bara & 0.06895 \\
°F & °C & $(T_F - 32) \times 5/9$ \\
ft & m & 0.3048 \\
mD & m² & $9.869 \times 10^{-16}$ \\
cP & Pa·s & 0.001 \\
STB & m³ & 0.1590 \\
MSCF & m³ & 28.32 \\
lb/ft³ & kg/m³ & 16.02 \\
\bottomrule
\end{tabular}
\caption{Common Unit Conversions}
\end{table}

\subsection{Typical Property Ranges}

\begin{table}[h]
\centering
\begin{tabular}{lll}
\toprule
\textbf{Property} & \textbf{Typical Range} & \textbf{Comments} \\
\midrule
API gravity & 10 - 50° & Light crude: 30-50°, Heavy: 10-25° \\
Gas gravity & 0.6 - 1.2 & Dry gas: 0.6-0.7, Wet gas: 0.7-0.9 \\
$R_s$ at $P_b$ & 100 - 3000 scf/stb & Depends on oil type and pressure \\
$B_o$ & 1.01 - 2.0 rb/stb & Increases with $R_s$ \\
$\mu_o$ & 0.5 - 100 cP & Light oil: 0.5-5, Heavy: 10-100+ \\
Z-factor & 0.7 - 1.2 & Lower at medium pressures \\
$B_g$ & 0.003 - 0.010 rcf/scf & Increases with T, decreases with P \\
$\mu_g$ & 0.01 - 0.03 cP & Increases with P and T \\
Permeability & 0.1 - 1000 mD & Tight: $<$1, Good: 100-1000 \\
Porosity & 0.05 - 0.35 & Low: 0.05-0.15, Good: 0.20-0.30 \\
\bottomrule
\end{tabular}
\caption{Typical Property Ranges for Petroleum Reservoirs}
\end{table}

\newpage

%=============================================================================
\section{Correlation Method Recommendations}
%=============================================================================

\subsection{Oil PVT Methods}

\begin{table}[h]
\centering
\begin{tabular}{llp{7cm}}
\toprule
\textbf{Property} & \textbf{Recommended} & \textbf{When to Use} \\
\midrule
Bubble Point & VALMC & Wide range of conditions, most accurate \\
 & STAN & Classic correlation, quick estimates \\
 & VELAR & Regional correlations, comparison \\
\midrule
Solution GOR & VELAR & Default choice, good accuracy \\
 & STAN & Compatibility with older studies \\
 & VALMC & Alternative method \\
\midrule
FVF & MCAIN & Most accurate, density-based \\
 & STAN & Simple correlation, fast \\
\midrule
Viscosity & BR & Industry standard, widely validated \\
\bottomrule
\end{tabular}
\caption{Recommended Correlations for Oil PVT}
\end{table}

\subsection{Gas PVT Methods}

\begin{table}[h]
\centering
\begin{tabular}{llp{7cm}}
\toprule
\textbf{Property} & \textbf{Recommended} & \textbf{When to Use} \\
\midrule
Z-Factor & DAK & Most accurate, industry standard \\
 & HY & Fast, good for most conditions \\
 & WYW & Newer correlation, comparison \\
 & BUR & High non-hydrocarbons, 100\% CO$_2$, H$_2$ \\
\midrule
Critical Props & PMC & Best accuracy, accounts for non-HCs \\
 & SUT & Classic method, compatibility \\
 & BUR & High non-hydrocarbon content \\
\midrule
Viscosity & LGE & Industry standard \\
\bottomrule
\end{tabular}
\caption{Recommended Correlations for Gas PVT}
\end{table}

\subsection{Relative Permeability Methods}

\begin{table}[h]
\centering
\begin{tabular}{llp{7cm}}
\toprule
\textbf{Method} & \textbf{Use For} & \textbf{Advantages} \\
\midrule
Corey & Quick estimates & Simple, fast, widely used \\
 & Standard cases & Few parameters, easy to understand \\
 & Initial models & Good starting point \\
\midrule
LET & History matching & Flexible curve shapes \\
 & Complex systems & Better fit to lab data \\
 & Final models & Independent endpoint control \\
\bottomrule
\end{tabular}
\caption{Relative Permeability Method Selection}
\end{table}

\newpage

%=============================================================================
\section{Tool Categories and Count}
%=============================================================================

\subsection{Available Tools by Category}

\begin{table}[h]
\centering
\begin{tabular}{clc}
\toprule
\textbf{Category} & \textbf{Description} & \textbf{Tools} \\
\midrule
\textbf{Oil PVT} & Oil property calculations & 17 \\
 & \small Bubble point, Rs, Bo, viscosity, density, etc. & \\
\midrule
\textbf{Gas PVT} & Gas property calculations & 11 \\
 & \small Z-factor, critical props, Bg, viscosity, etc. & \\
\midrule
\textbf{Inflow} & Well performance & 4 \\
 & \small Oil/gas rates for radial/linear flow & \\
\midrule
\textbf{Simulation} & Reservoir simulation support & 3 \\
 & \small Rel perm tables, aquifer functions, flash & \\
\midrule
\textbf{Brine} & Brine properties & 2 \\
 & \small CH$_4$ and CO$_2$ saturated brine & \\
\midrule
\textbf{Heterogeneity} & Layer analysis & 5 \\
 & \small Lorenz, beta, layer distributions & \\
\midrule
\textbf{Library} & Component properties & 1 \\
 & \small Critical properties for 100+ components & \\
\midrule
\textbf{Config} & Configuration resources & 4 \\
 & \small Units, methods, constants, help & \\
\midrule
\textbf{Geomechanics} & Wellbore stability \& stress & 27 \\
 & \small Stress analysis, breakouts, fracturing, sanding & \\
\midrule
\textbf{Total} & & \textbf{70} \\
\bottomrule
\end{tabular}
\caption{pyResToolbox MCP Tool Categories}
\end{table}

\subsection{Key Features}

\begin{itemize}
\item \textbf{Production Ready}: All tools tested and validated
\item \textbf{Field Units}: Consistent use of US oilfield units
\item \textbf{Array Support}: Calculate properties at multiple pressures simultaneously
\item \textbf{Multiple Methods}: Choose from various industry-standard correlations
\item \textbf{Type Safety}: Pydantic models ensure input validation
\item \textbf{Zero Configuration}: Works out of the box with Claude Desktop
\item \textbf{GPL-3.0 Licensed}: Free and open source
\end{itemize}

\newpage

%=============================================================================
\section{Usage Notes and Best Practices}
%=============================================================================

\subsection{General Guidelines}

\subsubsection{Input Validation}
\begin{itemize}
\item Always use absolute pressures (psia, not psig or barg)
\item Temperature must be in Fahrenheit, not Celsius
\item Check that input ranges are physically realistic
\item Verify that saturation endpoints sum correctly (e.g., $S_{wc} + S_{or} < 1.0$)
\end{itemize}

\subsubsection{Method Selection}
\begin{itemize}
\item Use recommended methods unless specific requirements exist
\item For bubble point: VALMC is most accurate for wide ranges
\item For Z-factor: DAK is industry standard, BUR for high non-HCs
\item For relative permeability: Corey for quick estimates, LET for accuracy
\end{itemize}

\subsubsection{Common Mistakes to Avoid}
\begin{itemize}
\item Using separator temperature instead of reservoir temperature
\item Confusing $R_{sb}$ (solution GOR at bubble point) with separator GOR
\item Not setting Vogel=True when $P_i < P_b$ (saturated reservoir)
\item Using wrong gas gravity (separator vs weighted average)
\item Not accounting for non-hydrocarbon components (H$_2$S, CO$_2$, N$_2$)
\item Forgetting to convert units from SI to Field Units
\end{itemize}

\subsection{PVT Calculation Workflow}

\subsubsection{Typical Oil PVT Workflow}
\begin{enumerate}
\item Calculate bubble point pressure ($P_b$) from $R_{sb}$, API, $T$, $\gamma_g$
\item Check if reservoir is undersaturated ($P_i > P_b$) or saturated ($P_i < P_b$)
\item Calculate $R_s(P)$ at desired pressures
\item Calculate $B_o(P, R_s)$ using McCain or Standing method
\item Calculate $\mu_o(P, R_s)$ using Beggs-Robinson
\item Calculate $\rho_o(B_o, R_s)$ using SWMH method
\item Calculate $c_o$ from numerical derivatives of $B_o$
\item Generate complete PVT table for simulation
\end{enumerate}

\subsubsection{Typical Gas PVT Workflow}
\begin{enumerate}
\item Calculate pseudo-critical properties ($T_c$, $P_c$) from $\gamma_g$ and impurities
\item Calculate Z-factor at desired pressures using DAK or BUR
\item Calculate $B_g(P, T, Z)$ from real gas equation
\item Calculate gas density $\rho_g(P, T, Z)$
\item Calculate gas viscosity $\mu_g(\rho_g, T)$ using LGE method
\item Calculate gas compressibility $c_g$ from Z-factor derivatives
\item Calculate pseudopressure $\psi(P)$ by integrating $P/(\mu_g Z)$
\end{enumerate}

\subsection{Well Performance Analysis}

\subsubsection{IPR Generation Steps}
\begin{enumerate}
\item Gather reservoir data: $P_i$, $P_b$, API, $T$, $h$, $k$, $S$, $r_e$, $r_w$
\item Calculate PVT properties at average pressure
\item Choose flow geometry: radial (vertical well) or linear (horizontal well)
\item For saturated reservoir ($P_i < P_b$), set Vogel=True
\item Calculate rates at multiple flowing pressures: $P_{wf}$ = 0 to $P_i$
\item Plot $q$ vs $P_{wf}$ to generate IPR curve
\item Identify AOF (absolute open flow = rate at $P_{wf}$ = 0)
\end{enumerate}

\subsection{Simulation Preparation}

\subsubsection{Relative Permeability Table Generation}
\begin{enumerate}
\item Choose table type: SWOF (water-oil), SGOF (gas-oil), or SGWFN (3-phase)
\item Choose correlation: Corey (simple) or LET (flexible)
\item Specify saturation endpoints: $S_{wc}$, $S_{or}$, $S_{org}$, $S_{gc}$
\item Set maximum relative permeabilities: $k_{rw}^{max}$, $k_{ro}^{max}$, $k_{rg}^{max}$
\item Define correlation parameters (Corey exponents or LET parameters)
\item Generate table with 20-50 saturation points
\item Validate endpoints and curves against core data
\item Export to ECLIPSE/CMG format for simulation
\end{enumerate}

\newpage

%=============================================================================
\section{References}
%=============================================================================

\subsection{Key Publications}

\subsubsection{Oil PVT Correlations}
\begin{itemize}
\item Standing, M.B. (1947). "A Pressure-Volume-Temperature Correlation for Mixtures of California Oils and Gases." \textit{Drilling and Production Practices}, API.
\item Valko, P.P. and McCain, W.D. (2003). "Reservoir Oil Bubblepoint Pressures Revisited; Solution Gas-Oil Ratios and Surface Gas Specific Gravities." \textit{Journal of Petroleum Science and Engineering}, Vol. 37, Issues 3-4, pp. 153-169. DOI: 10.1016/S0920-4105(02)00319-4.
\item Velarde, J., Blasingame, T.A., and McCain, W.D. (1997). "Correlation of Black Oil Properties at Pressures Below Bubble Point Pressure." \textit{SPE Annual Technical Conference and Exhibition}. (Note: Based on Velarde's 1996 Master's thesis at Texas A\&M University).
\item McCain, W.D., Spivey, J.P., and Lenn, C.P. (2011). \textit{Petroleum Reservoir Fluid Property Correlations}. PennWell Books.
\end{itemize}

\subsubsection{Gas PVT Correlations}
\begin{itemize}
\item Dranchuk, P.M. and Abou-Kassem, J.H. (1975). "Calculation of Z Factors for Natural Gases Using Equations of State." \textit{Journal of Canadian Petroleum Technology}.
\item Hall, K.R. and Yarborough, L. (1973). "A New Equation of State for Z-Factor Calculations." \textit{Oil and Gas Journal}.
\item Wang, X., Ye, F., and Wu, K. (2021). "A New Correlation for Calculating Gas Compressibility Factor." \textit{Energy Reports}, Vol. 8, Supplement 2, pp. 130-137. DOI: 10.1016/j.egyr.2021.11.029.
\item Burgoyne, M.W. (Santos Ltd), Nielsen, M.H., and Stanko, M. (Whitson AS) (2025). "A Universal, EOS-Based Correlation for Z-Factor, Viscosity and Enthalpy for Hydrocarbon and H$_2$, N$_2$, CO$_2$, H$_2$S Gas Mixtures." \textit{SPE Paper 229932-MS}. Presented at ADIPEC, Abu Dhabi, UAE, 3-6 November 2025. Supports high concentrations of non-hydrocarbons including 100\% CO$_2$ and H$_2$.
\item Lee, A.L., Gonzalez, M.H., and Eakin, B.E. (1966). "The Viscosity of Natural Gases." \textit{Journal of Petroleum Technology}.
\item Piper, L.D., McCain, W.D., and Corredor, J.H. (1993). "Compressibility Factors for Naturally Occurring Petroleum Gases." \textit{SPE Annual Technical Conference and Exhibition}, SPE-26668-MS, Houston, Texas, October 1993.
\item Sutton, R.P. (1985). "Compressibility Factors for High-Molecular-Weight Reservoir Gases." \textit{SPE Annual Technical Conference}.
\end{itemize}

\subsubsection{Well Performance}
\begin{itemize}
\item Darcy, H. (1856). \textit{Les Fontaines Publiques de la Ville de Dijon}. Victor Dalmont.
\item Vogel, J.V. (1968). "Inflow Performance Relationships for Solution-Gas Drive Wells." \textit{Journal of Petroleum Technology}.
\item Standing, M.B. (1971). "Concerning the Calculation of Inflow Performance of Wells Producing from Solution Gas Drive Reservoirs." \textit{Journal of Petroleum Technology}.
\end{itemize}

\subsubsection{Relative Permeability}
\begin{itemize}
\item Corey, A.T. (1954). "The Interrelation Between Gas and Oil Relative Permeabilities." \textit{Production Monthly}, Vol. 19, pp. 38-41.
\item Lomeland, F., Ebeltoft, E., and Thomas, W.H. (2005). "A New Versatile Relative Permeability Correlation." \textit{International Symposium of the Society of Core Analysts}.
\end{itemize}

\subsubsection{Brine Properties}
\begin{itemize}
\item McCain, W.D. (2011). \textit{Petroleum Reservoir Fluid Property Correlations}. PennWell Books, Chapter 4.
\item Spycher, N. and Pruess, K. (2010). "A Phase-Partitioning Model for CO$_2$-Brine Mixtures at Elevated Temperatures and Pressures: Application to CO$_2$-Enhanced Geothermal Systems." \textit{Transport in Porous Media}, Vol. 82, pp. 173-196. DOI: 10.1007/s11242-009-9425-y.
\item Garcia, J.E. (2001). "Density of Aqueous Solutions of CO$_2$." LBNL Report \#49023, October 2001. Also published in \textit{Fluid Phase Equilibria}.
\item Mao, S. and Duan, Z. (2009). "The Viscosity of Aqueous Alkali-Chloride Solutions up to 623 K, 1,000 bar, and High Ionic Strength." \textit{Chemical Geology}.
\item Islam, A.W. and Carlson, E.S. (2012). "Viscosity Models and Effects of Dissolved CO$_2$." \textit{Energy \& Fuels}, Vol. 26, No. 8, pp. 5330-5336. DOI: 10.1021/ef3006228.
\end{itemize}

\subsubsection{Reservoir Heterogeneity}
\begin{itemize}
\item Dykstra, H. and Parsons, R.L. (1950). "The Prediction of Oil Recovery by Waterflood." \textit{Secondary Recovery of Oil in the United States}. API.
\item Schmalz, J.P. and Rahme, H.D. (1950). "The Variation of Waterflood Performance with Variation in Permeability Profile." \textit{Producers Monthly}.
\item Lake, L.W. (1989). \textit{Enhanced Oil Recovery}. Prentice Hall.
\end{itemize}

\subsubsection{Geomechanics and Wellbore Stability}
\begin{itemize}
\item Zoback, M.D. (2007). \textit{Reservoir Geomechanics}. Cambridge University Press.
\item Fjaer, E., Holt, R.M., Horsrud, P., Raaen, A.M., and Risnes, R. (2008). \textit{Petroleum Related Rock Mechanics}, 2nd Edition. Elsevier.
\item Kirsch, G. (1898). "Die Theorie der Elastizität und die Bedürfnisse der Festigkeitslehre." \textit{Zeitschrift des Vereines deutscher Ingenieure}, Vol. 42, pp. 797-807.
\item Eaton, B.A. (1969). "Fracture Gradient Prediction and Its Application in Oilfield Operations." \textit{Journal of Petroleum Technology}, Vol. 21, No. 10, pp. 1353-1360.
\item Eaton, B.A. (1975). "The Equation for Geopressure Prediction from Well Logs." \textit{SPE 5544}, Fall Meeting of SPE-AIME, Dallas, Texas.
\item Bradley, W.B. (1979). "Failure of Inclined Boreholes." \textit{Journal of Energy Resources Technology}, Vol. 101, pp. 232-239.
\item Aadnoy, B.S. (1988). "Modeling of the Stability of Highly Inclined Boreholes in Anisotropic Rock Formations." \textit{SPE Drilling Engineering}, Vol. 3, No. 3, pp. 259-268.
\item Byerlee, J.D. (1978). "Friction of Rocks." \textit{Pure and Applied Geophysics}, Vol. 116, pp. 615-626.
\item McNally, G.H. (1987). "Estimation of Coal Measures Rock Strength Using Sonic and Neutron Logs." \textit{Geoexploration}, Vol. 24, pp. 381-395.
\item Horsrud, P. (2001). "Estimating Mechanical Properties of Shale from Empirical Correlations." \textit{SPE Drilling \& Completion}, Vol. 16, No. 2, pp. 68-73.
\item Chang, C., Zoback, M.D., and Khaksar, A. (2006). "Empirical Relations between Rock Strength and Physical Properties in Sedimentary Rocks." \textit{Journal of Petroleum Science and Engineering}, Vol. 51, pp. 223-237.
\item Vernik, L., Bruno, M., and Bovberg, C. (1993). "Empirical Relations between Compressive Strength and Porosity of Siliciclastic Rocks." \textit{International Journal of Rock Mechanics and Mining Sciences}, Vol. 30, No. 7, pp. 677-680.
\item Al-Ajmi, A.M. and Zimmerman, R.W. (2005). "Relation between the Mogi and the Coulomb Failure Criteria." \textit{International Journal of Rock Mechanics and Mining Sciences}, Vol. 42, pp. 431-439.
\end{itemize}

\subsection{Software and Tools}
\begin{itemize}
\item Burgoyne, M.W. (2024). \textit{pyResToolbox: A Collection of Reservoir Engineering Utilities}. GitHub: \url{https://github.com/mwburgoyne/pyResToolbox}
\item Anthropic (2024). \textit{Model Context Protocol}. \url{https://modelcontextprotocol.io/}
\item Serrao, G. (2024). \textit{pyResToolbox MCP Server}. GitHub: \url{https://github.com/gabrielserrao/pyrestoolbox-mcp}
\end{itemize}

\newpage

%=============================================================================
\section{Appendix: Nomenclature}
%=============================================================================

\subsection{Primary Symbols}

\begin{longtable}{llp{8cm}}
\toprule
\textbf{Symbol} & \textbf{Units} & \textbf{Description} \\
\midrule
\endfirsthead
\toprule
\textbf{Symbol} & \textbf{Units} & \textbf{Description} \\
\midrule
\endhead

% Oil Properties
$\text{API}$ & degrees & Oil API gravity \\
$B_o$ & rb/stb & Oil formation volume factor \\
$c_o$ & 1/psi & Oil compressibility \\
$\mu_o$ & cP & Oil viscosity \\
$\rho_o$ & lb/ft³ & Oil density \\
$R_s$ & scf/stb & Solution gas-oil ratio \\
$R_{sb}$ & scf/stb & Solution GOR at bubble point \\
$P_b$ & psia & Bubble point pressure \\
$\gamma_o$ & - & Oil specific gravity (water=1) \\

% Gas Properties
$B_g$ & rcf/scf & Gas formation volume factor \\
$c_g$ & 1/psi & Gas compressibility \\
$\mu_g$ & cP & Gas viscosity \\
$\rho_g$ & lb/ft³ & Gas density \\
$Z$ & - & Gas compressibility factor (Z-factor) \\
$\gamma_g$ & - & Gas specific gravity (air=1) \\
$P_c$ & psia & Pseudo-critical pressure \\
$T_c$ & °R & Pseudo-critical temperature \\
$P_{pr}$ & - & Pseudo-reduced pressure \\
$T_{pr}$ & - & Pseudo-reduced temperature \\
$\psi$ & psia²/cP & Gas pseudopressure \\

% Brine Properties
$B_w$ & rb/stb & Water/brine formation volume factor \\
$c_w$ & 1/psi & Brine compressibility \\
$\mu_w$ & cP & Brine viscosity \\
$\rho_w$ & lb/ft³ & Brine density \\
$R_w$ & scf/stb & Gas solubility in brine \\
$w_t$ & wt\% & Salinity (weight percent NaCl) \\

% Reservoir Properties
$k$ & mD & Permeability \\
$h$ & ft & Pay thickness (net) \\
$\phi$ & fraction & Porosity \\
$S_w$ & fraction & Water saturation \\
$S_o$ & fraction & Oil saturation \\
$S_g$ & fraction & Gas saturation \\
$S_{wc}$ & fraction & Connate water saturation \\
$S_{or}$ & fraction & Residual oil saturation \\

% Well/Flow Properties
$P$ & psia & Pressure \\
$T$ & °F & Temperature \\
$P_i$ & psia & Initial/reservoir pressure \\
$P_{wf}$ & psia & Wellbore flowing pressure \\
$q_o$ & STB/day & Oil production rate \\
$q_g$ & MSCF/day & Gas production rate \\
$r_e$ & ft & Drainage radius \\
$r_w$ & ft & Wellbore radius \\
$S$ & - & Skin factor \\
$L$ & ft & Horizontal well length \\

% Relative Permeability
$k_{rw}$ & - & Water relative permeability \\
$k_{ro}$ & - & Oil relative permeability \\
$k_{rg}$ & - & Gas relative permeability \\
$S^*$ & - & Normalized saturation \\

% Heterogeneity
$L$ & - & Lorenz coefficient \\
$\beta$ & - & Dykstra-Parsons coefficient \\

% Non-Hydrocarbons
$y_{CO_2}$ & fraction & CO$_2$ mole fraction \\
$y_{H_2S}$ & fraction & H$_2$S mole fraction \\
$y_{N_2}$ & fraction & N$_2$ mole fraction \\

% Geomechanics Stresses
$\sigma_v$ & psi & Vertical (overburden) stress \\
$\sigma_H$ & psi & Maximum horizontal stress \\
$\sigma_h$ & psi & Minimum horizontal stress \\
$\sigma_\theta$ & psi & Tangential (hoop) stress \\
$\sigma_r$ & psi & Radial stress \\
$\sigma'$ & psi & Effective stress \\

% Geomechanics Properties
$\alpha$ & - & Biot coefficient \\
$E$ & psi & Young's modulus \\
$\nu$ & - & Poisson's ratio \\
$K$ & psi & Bulk modulus \\
$G$ & psi & Shear modulus \\
$c$ & psi & Cohesion \\
$\phi$ & degrees & Internal friction angle \\
UCS & psi & Unconfined compressive strength \\
$T_0$ & psi & Tensile strength \\

% Geomechanics Failure/Stability
$\tau$ & psi & Shear stress \\
$\sigma_n$ & psi & Normal stress \\
$\mu$ & - & Fault friction coefficient \\
$T_s$ & - & Slip tendency \\
$T_d$ & - & Dilation tendency \\
CFS & psi & Coulomb Failure Stress \\

% Geomechanics Thermal
$\alpha_T$ & 1/°F & Thermal expansion coefficient \\
$\Delta T$ & °F & Temperature change \\

\bottomrule
\caption{Nomenclature - Primary Symbols}
\end{longtable}

\subsection{Subscripts}

\begin{table}[h]
\centering
\begin{tabular}{ll}
\toprule
\textbf{Subscript} & \textbf{Meaning} \\
\midrule
$o$ & Oil phase \\
$g$ & Gas phase \\
$w$ & Water/brine phase \\
$b$ & At bubble point \\
$i$ & Initial conditions \\
$wf$ & Wellbore flowing \\
$c$ & Critical or connate \\
$r$ & Relative or residual \\
$pr$ & Pseudo-reduced \\
$D$ & Dimensionless \\
$sc$ & Standard conditions \\
$ST$ & Stock tank conditions \\
$sp$ & Separator conditions \\
\bottomrule
\end{tabular}
\caption{Common Subscripts}
\end{table}

\newpage

%=============================================================================
\section{Index of Formulas}
%=============================================================================

\subsection{By Property}

\textbf{Oil Properties}
\begin{itemize}
\item Bubble Point Pressure ($P_b$): Standing (p.4), Valko-McCain (p.4), Velarde (p.4)
\item Solution GOR ($R_s$): Standing (p.5), Valko-McCain, Velarde
\item Formation Volume Factor ($B_o$): McCain (p.6), Standing (p.6)
\item Viscosity ($\mu_o$): Beggs-Robinson (p.7)
\item Density ($\rho_o$): SWMH (p.7)
\item Compressibility ($c_o$): Explicit method (p.8)
\item API-SG Conversions (p.8)
\end{itemize}

\textbf{Gas Properties}
\begin{itemize}
\item Z-Factor: DAK (p.9), Hall-Yarborough (p.10)
\item Critical Properties: PMC (p.10), Sutton, BUR
\item Formation Volume Factor ($B_g$): Real gas equation (p.11)
\item Viscosity ($\mu_g$): Lee-Gonzalez-Eakin (p.11)
\item Density ($\rho_g$): Ideal gas law corrected (p.11)
\item Compressibility ($c_g$): From Z-factor (p.12)
\item Pseudopressure ($\psi$): Integral formulation (p.12)
\end{itemize}

\textbf{Well Performance}
\begin{itemize}
\item Oil Rate - Radial: Darcy's law (p.13), Vogel IPR (p.13)
\item Oil Rate - Linear: Darcy's law (p.14)
\item Gas Rate - Radial: Pseudopressure formulation (p.14)
\item Gas Rate - Linear: Pseudopressure formulation (p.14)
\end{itemize}

\textbf{Relative Permeability}
\begin{itemize}
\item Corey Correlation: Normalized saturation (p.15), Kr functions (p.15)
\item LET Correlation: Three-parameter model (p.16)
\end{itemize}

\textbf{Brine Properties}
\begin{itemize}
\item Density ($\rho_w$): Pressure-temperature-salinity (p.17)
\item Viscosity ($\mu_w$): Salinity correction (p.17)
\item Compressibility ($c_w$): From FVF derivative (p.18)
\item Formation Volume Factor ($B_w$): Three-effect model (p.18)
\item Gas Solubility ($R_w$): Henry's Law (p.18)
\end{itemize}

\textbf{Heterogeneity}
\begin{itemize}
\item Lorenz Coefficient: Integral formulation (p.19)
\item Dykstra-Parsons Coefficient: Percentile definition (p.19)
\item Lorenz-Beta Conversion: Approximate relations (p.20)
\item Layer Distribution: Log-normal model (p.20)
\end{itemize}

\textbf{Geomechanics}
\begin{itemize}
\item Vertical Stress: Overburden calculation
\item Pore Pressure: Eaton method (sonic/resistivity)
\item Effective Stress: Terzaghi and Biot formulations
\item Horizontal Stress: Poroelastic model, stress regimes
\item Elastic Moduli: E, K, G, $\lambda$ conversions
\item Rock Strength: Mohr-Coulomb criterion
\item Fracture Gradient: Eaton method, EMW conversion
\item Wellbore Stress: Kirsch equations (hoop/radial)
\item Breakout Width: Failure initiation angle
\item Breakdown Pressure: Hydraulic fracturing initiation
\item Mud Weight Window: Kick, collapse, lost circulation bounds
\item Reservoir Compaction: Uniaxial strain model
\item Stress Path: Depletion/injection effects
\item Stress Polygon: Frictional equilibrium constraints
\item Fault Stability: Coulomb failure stress, slip tendency
\item Thermal Stress: Temperature-induced stress changes
\item UCS from Logs: McNally, Horsrud, Chang, Vernik correlations
\item Critical Drawdown: Sand production prediction
\item Failure Criteria: Mohr-Coulomb, Drucker-Prager, Mogi-Coulomb
\end{itemize}

\textbf{Other}
\begin{itemize}
\item Van Everdingen-Hurst Aquifer: Water influx
\item Rachford-Rice Flash: Phase equilibrium
\end{itemize}

\vspace{2cm}

\begin{center}
\rule{\linewidth}{0.5pt}
\vspace{0.5cm}

\Large\textbf{End of Formula Reference Guide}

\vspace{0.5cm}

\textbf{Compiled by:} Gabriel Serrão Seabra

\vspace{0.5cm}

For the latest version and updates, visit:\\
\url{https://github.com/gabrielserrao/pyrestoolbox-mcp}

\vspace{0.5cm}

\textbf{Built on pyResToolbox:}\\
Original library by Mark W. Burgoyne\\
\url{https://github.com/mwburgoyne/pyResToolbox}

\vspace{0.5cm}

\textit{Remember: Always verify formulas with original sources before professional use.}

\vspace{0.5cm}
\rule{\linewidth}{0.5pt}
\end{center}

\end{document}
